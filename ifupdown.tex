
\documentclass{article}
\usepackage{graphicx}
\usepackage{noweb}
\usepackage{tikz}
\usetikzlibrary{calc}
\usetikzlibrary{positioning}
\pagestyle{noweb}
\noweboptions{smallcode,hideunuseddefs}
\begin{document}
@

\def\nwendcode{\endtrivlist\endgroup}
\let\nwdocspar=\relax

\title{ Interface Tools\thanks{
Copyright \copyright\ 1999--2007 Anthony Towns. This program is free
software; you can redistribute it and/or modify it under the terms of
the GNU General Public License as published by the Free Software
Foundation; either version 2 of the License, or (at your option) any
later version.  
}}

\author{ Anthony Towns \\ { \tt aj@azure.humbug.org.au } }

\pagenumbering{roman}

\maketitle
\tableofcontents

\vfill
\pagebreak
\pagenumbering{arabic}

\section{Introduction}

This source defines the commands [[ifup]], [[ifdown]], and [[ifquery]], used
to manipulate interfaces in an easily controllable manner.

\subsection{Assumed Knowledge}

The reader is assumed to have knowledge of the C \cite{K&R} and Perl
\cite{camel} programming languages in a Unix environment \cite{StevensUnix}.
A cursory understanding of network administration on the appropriate
platform is also assumed, along with access to the relevant manual
pages as necessary.

This source has been written as a literate program using the [[noweb]]
\cite{wwwnoweb} tool suite, and typeset using \LaTeX\ \cite{latex}.

\subsection{Program Structure}

We shall decompose this program into four main areas of functionality:
compile-time configuration, run-time configuration, execution, and the
overall driver.

Compile-time configuration will deal with differing available address
families (IP vs IPX vs IPv6, and so on), and the differing methods of
enabling and disabling interfaces configured for each family. This will
be implemented using the [[addrfam]] module, and various [[.defn]] files,
for the address family definitions.

Run-time configuration will deal with determining the local setup
based on the file [[/etc/network/interfaces]], and producing a data
structure encapsulating these details. This will be implemented in the
[[config]] module.

Execution will deal with issues relating to working out exactly which
commands to run based on a somewhat abstract description from the
compile-time configuration and the details determined at
run-time. This will be dealt with in the [[execute]] module.

The remaining work --- argument parsing, error reporting, and,
essentially, putting all the pieces together --- is done by the
[[main]] module.

The following diagram gives a brief idea of the information and control
flow amongst the modules.

\def\vsep{2.5cm}
\def\hsep{1.4cm}
\begin{figure}
\centering
\begin{tikzpicture}[every text node part/.style={text centered}]
    \tikzstyle{block}=[rectangle,draw, text width=2cm, minimum height=2cm]
    \tikzstyle{big}=[minimum height=4.5cm]
    \tikzstyle{annot}=[above left]

    \node[block] (ip) at (0,0) {IP address family};
    \node[annot] at (ip.south east) {\small inet.defn};

    \node[block] (ipx) at (0,-\vsep) {IPX address
    family};
    \node[annot] at (ipx.south east) {\small inet.defn};

    \node[block,big] (addrfam) at (\hsep*2,-\vsep/2) {Compile time configuration};
    \node[annot] at (addrfam.south east) {\small addrfam.c};

    \node[block,big] (config) at (\hsep*4,-\vsep/2) {Runtime configuration};
    \node[annot] at (config.south east) {\small config.c};

    \node[block] (exec) at (\hsep*6,0) {Execution};
    \node[annot] at (exec.south east) {\small execute.c};

    \node[block] (main) at (\hsep*6,-\vsep) {Driver};
    \node[annot] at (main.south east) {\small main.c};

    \draw[-stealth] (ip.east) -- (addrfam.west |- ip);
    \draw[-stealth] (ipx.east) -- (addrfam.west |- ipx);
    \draw[-stealth] (addrfam) -- (config);
    \draw[stealth-] (exec.west) -- (config.east |- exec);
    \draw[stealth-] (main.west) -- (config.east |- main);
    \draw[-stealth] (main) -- (exec);

    \node[below=1.2cm of config.south] (mainf) {main()};

    \draw[-stealth, rounded corners=0.3cm] (mainf) -- +(0,1cm) -| (main.south);
\end{tikzpicture}
\end{figure}


Much of the information sharing will be done by defining and filling
in some data structures and allowing the other modules to just access
that information directly. Rather than hiding the information itself,
most of our modules simply attempt to hide how that information was
originally written. Because of this, we shall find that these modules are
too closely linked to be completely separated in a convenient manner,
so they will all make use of a single header file for each other's
structure definitions, exported interfaces and so on.

<<header.h>>=
#ifndef HEADER_H
#define HEADER_H

#include <stdbool.h>

<<type definitions>>
<<function type definitions>>
<<structure definitions>>
<<constant definitions>>
<<exported symbols>>
<<address family declarations>>

#endif /* HEADER_H */
@ 

\section{The Build System}

We shall begin with the template for the Makefile we shall use.

<<Makefile>>=
<<make options>>
ARCH ?= linux

BASEDIR ?= $(DESTDIR)

CFILES := addrfam.c execute.c config.c main.c arch$(ARCH).c
HFILES := header.h arch$(ARCH).h
PERLFILES := defn2c.pl defn2man.pl
DEFNFILES := inet.defn ipx.defn inet6.defn can.defn

OBJ := main.o addrfam.o execute.o config.o \
	$(patsubst %.defn,%.o,$(DEFNFILES)) arch$(ARCH).o meta.o link.o

MAN := $(patsubst %.defn,%.man,$(DEFNFILES))
DEFNFILES += meta.defn link.defn

default : executables
all : executables docs

executables : ifup ifdown ifquery ifup.8 ifdown.8 ifquery.8 interfaces.5
docs : ifupdown.ps.gz ifup.8.ps.gz interfaces.5.ps.gz ifupdown.pdf

.PHONY : executables 
<<phony targets>>
<<executable targets>>
<<manpage targets>>
<<extra dependencies>>
<<implicit rules>>

<<generated dependency inclusion>>
@ 

We shall build exactly three executables, [[ifup]], [[ifdown]] and
[[ifquery]], which will in truth simply be three names for a single binary,
albeit with different functionality.

<<executable targets>>=
ifup: $(OBJ)
	$(CC) $(CFLAGS) $^ $(LDFLAGS) $(OUTPUT_OPTION)

ifdown: ifup
	ln -sf ifup ifdown

ifquery: ifup
	ln -sf ifup ifquery
@ 

All three of these executables have a manpage. Since they're actually the
same executable, what could be more appropriate than them having the
same manpage too?

<<manpage targets>>=
interfaces.5: interfaces.5.pre $(MAN)
	sed $(foreach man,$(MAN),-e '/^##ADDRESSFAM##$$/r $(man)') \
	     -e '/^##ADDRESSFAM##$$/d' < $< > $@	

ifdown.8 ifquery.8: ifup.8
	ln -sf $< $@

%.5.ps: %.5
	groff -mandoc -Tps $< > $@
%.8.ps: %.8
	groff -mandoc -Tps $< > $@
@ 

Further, for convenience, we'll make use of two phony targets, [[clean]],
[[clobber]] and [[distclean]], which will delete working files, everything
that can be rebuilt with a [[make]] command, and everything that can be
rebuilt at all, respectively.

<<phony targets>>=
.PHONY : clean clobber

install :
	install -m 0755 -d     ${BASEDIR}/sbin
	install -m 0755 ifup   ${BASEDIR}/sbin
	ln ${BASEDIR}/sbin/ifup ${BASEDIR}/sbin/ifdown	
	ln ${BASEDIR}/sbin/ifup ${BASEDIR}/sbin/ifquery

clean :
	rm -f *.aux *.toc *.log *.bbl *.blg *.ps *.eps *.pdf
	rm -f *.o *.d $(patsubst %.defn,%.c,$(DEFNFILES)) *~
	rm -f $(patsubst %.defn,%.man,$(DEFNFILES))
	rm -f ifup ifdown ifquery interfaces.5 ifdown.8 ifquery.8
	rm -f ifupdown.dvi *.ps{,.gz}

clobber : clean
	rm -f ifupdown.tex $(PERLFILES) $(CFILES) $(HFILES) $(DEFNFILES) arch*

distclean : clobber
	rm -f makecdep.sh makenwdep.sh Makefile
@ 

We have some fairly standard rules to build the printed version of the
source code using \LaTeX\ that are, unfortunately, not included in
[[make(1)]]'s builtin rules, so we'll note them here.

<<implicit rules>>=
%.tex : %.nw
	noweave -delay -index -latex $< >$@

%.bbl : %.tex biblio.bib
	latex $<
	bibtex $(basename $<)

%.dvi : %.tex %.bbl
	latex $<
	latex $<

%.pdf : %.tex %.bbl
	pdflatex $<
	pdflatex $<

%.ps : %.dvi
	dvips -o $@ $<

%.gz : %
	gzip --best --stdout $< >$@
@ 

Additionally, some of [[make]]'s builtin rules are fairly
conservative, so we'll encourage it to use a more entertaining method
of compiling source code.

<<make options>>=
VERSION ?= 0.7
CFLAGS ?= -Wall -W -g -O2 -D'IFUPDOWN_VERSION="$(VERSION)"'
@ 

\subsection{Graphics}

We include a few graphics (made using dia) in this document. We have to
express these fairly explicitly, unfortunately.

<<implicit rules>>=
%.eps : %.dia
	dia --nosplash -e $@ $<

%.pdf : %.eps
	gs -q -sDEVICE=pdfwrite -dNOPAUSE -sOutputFile=$@ - < $<
@

<<extra dependencies>>=
ifupdown.dvi: modules.eps execution.eps
ifupdown.ps: modules.eps execution.eps
ifupdown.pdf: modules.pdf execution.pdf
@

\subsection{Automatic Dependencies}

To build the system, we'll make use of some techniques discussed in
\cite{recursivemake} for determining dependencies. Namely, a number
of files will have an associated [[.d]] file containing dynamically
determined dependency information. The first such file we will construct
is the dependency information for [[noweb]] source files, which can be
identified by the [[.nw]] extension.

<<implicit rules>>=
%.d: %.nw makenwdep.sh
	./makenwdep.sh $< > $@
@

To construct the dependency information, we may use the [[noroots(1)]]
command to determine the \emph{root chunks} in the [[noweb]] source
(stripping the unwanted [[<<]] and [[>>]] markers as we go, and
denoting that in such a way that [[noweb]] doesn't mistakenly think
the [[sed]] command is a chunk reference itself), and then noting down
appropriate commands to construct the target.

<<makenwdep.sh>>=
<<parse makenwdep arguments>>

noroots $FILE | sed 's/^<''<\(.*\)>>$/\1/' |
	while read chunk; do
		<<output dependency info for [[$chunk]]>>
	done
@ 

Our dependency information is straightforward. To construct a file from
[[noweb]] source, we simply need to run [[notangle(1)]] over it. We add
a couple of extra tweaks in order to only update files that were actually
changed (the [[cpif(1)]] call), and to handle tabs properly.

We also need some extra things to take care of particular types of files.
In particular its important to have our scripts marked executable, so we
can use them as part of the build process itself, and it's also important
to have the dependency information for our C files (which are dealt with
next) included at some point.

<<output dependency info for [[$chunk]]>>=
printf "%s : %s\n" "$chunk" "$FILE"
case $chunk in
	*.pl|*.sh)
		printf "\tnotangle -R\$@ \$< >\$@\n"
		printf "\tchmod 755 %s\n" "$chunk"
		;;
	*.c)
		printf "\tnotangle -L -R\$@ \$< | cpif \$@\n"
		printf "include ${chunk%.c}.d\n"
		;;
	*.h)
		printf "\tnotangle -L -R\$@ \$< | cpif \$@\n"
		;;
	*)
		printf "\tnotangle -t8 -R\$@ $< >\$@\n"
		;;
esac
@ 

Finally, our fairly primitive argument parsing is simply:

<<parse makenwdep arguments>>=
FILE=$1

if [ "$FILE" = "" -o ! -f "$FILE" ]; then
	echo "Please specify a .nw file"
	exit 1
fi
@ 

We have a related system for object files generated from C source
code. Since each object file depends not only on its source, but also
the headers included in that source, we generate a [[.d]] file indicating
exactly which headers need to be checked.

<<implicit rules>>=
%.d: %.c makecdep.sh
	./makecdep.sh $< > $@
@ 

We can do this using [[gcc(1)]]'s convenient [[-MM -MG]] options,
which do exactly this, with the added proviso that the [[.d]] file
itself can possibly depend on any of the header files being modified
(and, in particular, [[#include]] lines being added or deleted).

<<makecdep.sh>>=
#!/bin/sh
<<parse makecdep arguments>>

gcc -MM -MG $FILE |
  sed -e 's@^\(.*\)\.o:@\1.o \1.d:@'
@ 

\emph{Deja vu}, anyone?

<<parse makecdep arguments>>= 
FILE=$1
if [ "$FILE" = "" -o ! -f "$FILE" ]; then
	echo "Please specify a .c file"
	exit 1
fi
@

To include the generated dependencies in [[Makefile]], we have to
be a bit careful.  The problem here is that they should not be rebuild
when merely the cleaning of the source tree is asked for.  Any targets
ending in [[clean]], plus the [[clobber]] target prevent the inclusion
of the generated dependencies.

Unfortunately, [[make]] doesn't allow logical combinations within
[[ifeq]] and friends, so we have to simulate this.

<<generated dependency inclusion>>=
include-deps := YES
ifneq "" "$(filter %clean,$(MAKECMDGOALS))"
include-deps := NO
endif
ifeq "clobber" "$(MAKECMDGOALS)"
include-deps := NO
endif
@

Finally, include the dependency information:

<<generated dependency inclusion>>=
ifeq "$(strip $(include-deps))" "YES"
include ifupdown.d
endif
@

\section{Compile Time Configuration}

At compile time we need to determine all the possible address families
that may be used, and all the methods of setting up interfaces for
those address families, along with the various possible options
affecting each method.

Our key definition at this point is that of the [[address_family]]
structure, which encapsulates all the compile time information about
each address family.

<<type definitions>>=
typedef struct address_family address_family;
@ 

<<structure definitions>>=
struct address_family {
	char *name;
	int n_methods;
	method *method;
};
@ 

Each defined address family will be included in the [[addr_fams]]
array, which becomes the \emph{raison d'\^etre} of the [[addrfam]]
module.

<<exported symbols>>=
extern address_family *addr_fams[];
@ 

Each address family incorporates a number of methods, which
encapsulate various ways of configuring an interface for a particular
address family. There are three components of a method: two sets of
commands to bring an interface up and down, a number of options for
the commands, and a set of conversion functions to allow programmatic
manipulation of the options.

<<type definitions>>=
typedef struct method method;
@ 

<<structure definitions>>=
struct method {
	char *name;
	command_set *up, *down;
	conversion *conversions;
	option_default *defaults;
};
@ 

Each command set is implemented as a single function, accepting two
parameters: the definitions of the interface the commands should deal
with, and the function that should be used to execute them. See the
[[execute]] module for more details.

<<function type definitions>>=
typedef int (execfn)(char *command);
typedef int (command_set)(interface_defn *ifd, execfn *e);
@ 

Each conversion is implemented as a mapping from an option name, to a
function that updates that option's value or adds a new variable which
value is the result of conversion, as follows:

<<type definitions>>=
typedef struct conversion conversion;
typedef struct option_default option_default;
@

<<structure definitions>>=
struct conversion {
	char *option;
	char *newoption;
	void (*fn)(interface_defn *, char **, int, char **);
	int argc;
	char ** argv;
};

struct option_default {
	char *option;
	char *value;
};
@

As our compile-time configuration is done at, well, compile-time, there
is little need for functions in the actual module, and we can make do with
a single exported array.

<<addrfam.c>>=
#include <stdlib.h>
#include "header.h"

address_family *addr_fams[] = {
	<<address family references>>
	NULL
};
@ 

\subsection{Generating C Code}

Unfortunately, while the [[.defn]] representation is reasonably
convenient for human use, it's less convenient for a compiler. As
such, at build time, we will build a single structure of type
[[address_family]] in a separate module, and reference that from
[[addrfam.c]].

Naturally, we'll use a [[perl]] script to convert [[.defn]] files to C
code.

<<implicit rules>>=
%.c : %.defn defn2c.pl
	./defn2c.pl $< > $@
@ 

The functionality of our program is pretty basic: read from the file
specified as the argument, output to [[stdout]]; and correspondingly
the structure of the program is similarly simple. We make use of a
couple of global variables, a few helpful subroutines, and then build
our C program.

<<defn2c.pl>>=
#!/usr/bin/perl -w

use strict;

<<determine the target architecture>>

# declarations
<<defn2c variables>>

# subroutines
<<defn2c subroutines>>

# main code
<<output headers for address family>>
<<parse [[.defn]] file and output intermediate structures>>
<<output address family data structure>>
@ 

First of all, we determine the target architecture by calling [[dpkg-architecture]] and stripping the trailing newline:
<<determine the target architecture>>=
my $DEB_HOST_ARCH_OS = `dpkg-architecture -qDEB_HOST_ARCH_OS`;

$DEB_HOST_ARCH_OS =~ s/\n//;
@ 

Clearly we need to reference some of the data structures we defined
above, so we can begin with the rather trivial:

<<output headers for address family>>=
print "#include <stddef.h>\n";
print "#include \"header.h\"\n\n\n";
@ 

The overall purpose of the C code we're trying to construct is to
define a structure of type [[address_family]], and have it externally
visible. So we'd like to declare a statically-initialized structure of
this type, and be done with it.

To do this, however, we need to have some way of uniquely identifying
the structure to avoid naming conflicts. We'll do this by assuming we
have a previously initialized variable, [[$address_family]], and that
names of the form [[addr_foo]] won't be used elsewhere.

<<defn2c variables>>=
my $address_family = "";
@

We also need to reference an array of pointers to this address
family's [[method]]s, which will have to be initialized separately.
We'll assume that previous code will create such a structure, and call
it (imaginatively) [[methods]].

<<output address family data structure>>=
<<output [[methods]] data structure>>

print <<EOF;
address_family addr_${address_family} = {
	"$address_family",
	sizeof(methods)/sizeof(struct method),
	methods
};
EOF
@ 

Our [[methods]] array will be a little more complicated to
construct. The first difficulty is that it will actually require some
state from having read the [[.defn]] file. To handle this, we'll
introduce a hash [[%methods]] that has the value [[1]] for each method
from the [[.defn]] file. We use a hash instead of a list because it
makes some error checking more convenient later, and since order
doesn't particularly matter.

<<defn2c variables>>=
my %methods = ();
my %ourmethods = ();
@

We'll use standard names, such as [[foo_up]], [[foo_down]], and
[[foo_conv]] for the various elements of each method which cannot
be defined inline. The two functions and the array will be declared
static to avoid name conflicts.

<<output [[methods]] data structure>>=
print "static method methods[] = {\n";
%ourmethods = %methods if (our_arch());
my $method;
foreach $method (keys %ourmethods) {
	print <<EOF;
	{
		"$method",
		_${method}_up, _${method}_down,
		_${method}_conv, _${method}_default
	},
EOF
}
print "};\n\n";
@ 

In a reasonably obvious manner we can then proceed to process the
[[.defn]] file to initialize the aforementioned variables, and to
output the method-specific functions and array. We'll begin by
defining a variable to keep track of the current line.

<<defn2c variables>>=
my $line = "";
@ 

Our semantics for this variable will be basically that it contains a
valid, meaningful input line. It won't be blank, or contain comments,
and if there aren't any more lines to be read, it will evaluate to
[[false]]. In order to keep these semantics, we'll use the subroutine
[[nextline]] to update it. (Since we'll later find we'll want to reuse
this subroutine, we'll keep it in a chunk of its own)

We need to extract from the [[.defn]] file the only set of methods that
corresponds to our target architecture. To perform this, we need to keep
the current architecture name in a variable which we define globally:

<<defn2c variables>>=
my $arch = "";
@ 

<<defn2c subroutines>>=
<<[[nextline]] subroutine>>
<<[[our_arch]] function>>
@ 

<<[[nextline]] subroutine>>=
sub nextline {
	$line = <>;
	while($line and ($line =~ /^#/ or $line =~ /^\s*$/)) {
		$line = <>;
	}
	if (!$line) { return 0; }
	chomp $line;
	while ($line =~ m/^(.*)\\$/) {
		my $addon = <>;
		chomp $addon;
		$line = $1 . $addon;
	}
	return 1;
}
@ 

We accept any valid Debian OS identifier (that is, [[DEB_HOST_ARCH_OS]]) as
the architecture name, as well as the literal string `[[any]]' which means
the definition file is functional on every architecture.

<<[[our_arch]] function>>=
sub our_arch {
    return ($arch eq $DEB_HOST_ARCH_OS) || ($arch eq "any")
}
@ 

Our high-level logic then looks basically like:

<<parse [[.defn]] file and output intermediate structures>>=
nextline;
while($line) {
	<<parse a top-level section and output intermediate structures>>

	# ...otherwise
	die("Unknown command \"$line\"");
}
@

To make all this stuff easier, we'll use a `matching' function to help
with parsing lines: basically, given a line, a command, and possibly
an indentation prefix (eg [["    "]]). (As with [[nextline]], this will
also be useful to reuse, hence it has its own pair of chunks, too)

<<defn2c variables>>=
my $match = "";
@ 

<<defn2c subroutines>>=
<<[[match]] subroutine>>
@ 

<<[[match]] subroutine>>=
sub match {
	my $line = $_[0];
	my $cmd = "$_[1]" ? "$_[1]\\b\\s*" : "";;
	my $indentexp = (@_ == 3) ? "$_[2]\\s+" : "";

	if ($line =~ /^${indentexp}${cmd}(([^\s](.*[^\s])?)?)\s*$/) {
		$match = $1;
		return 1;
	} else {
		return 0;
	} 
}
@ 

Okay. So, the first line we expect to see is the name of the address
family we're defining.

<<parse a top-level section and output intermediate structures>>=
if (match($line, "address_family")) {
	get_address_family $match;
	next;
}
@ 

This is, as you'd imagine, pretty simple to deal with. We just need to
store the address family's name, and move on to the next line.

<<defn2c subroutines>>=
sub get_address_family {
	$address_family = $_[0] if ($address_family eq "");
	nextline;
}
@ 

Which brings us to determining the architecture.

<<parse a top-level section and output intermediate structures>>=
if (match($line, "architecture")) {
	get_architecture $match;
	next;
}
@ 

You'd never guess what, but it's just as easy as the address family thing
was.

<<defn2c subroutines>>=
sub get_architecture {
        %ourmethods = %methods if (our_arch());
	$arch = $_[0];
	if (!our_arch) {
		%methods = ();
	} else {
		print "#include \"arch${DEB_HOST_ARCH_OS}.h\"\n\n\n";
	}
	nextline;
}
@ 

After each [[architecture]] keyword occurrence, we drop all the methods
defined for the previous architecture.

Next, we actually create the functions and array for each method.

<<parse a top-level section and output intermediate structures>>=
if (match($line, "method")) {
	if (our_arch()) {
		get_method $match;
	} else {
		skip_section;
	}
	next;
}
@

The basic premise is to check for each of our options in a given
order: if they don't match, then we can presume they don't exist ---
any errors will be reported when the main function finds something
weird going on. All we really have to take care of so far is ensuring
an appropriate level of indentation, and that we're not defining the
same method twice.

<<defn2c subroutines>>=
sub get_method {
	my $method = $_[0];
	my $indent = ($line =~ /(\s*)[^\s]/) ? $1 : "";
	my @options = ();
	my @variables = ();

	die "Duplicate method $method\n" if ($methods{$method}++);

	nextline;
	<<output code for description>>
	<<output code for options list>>
	<<output code for conversions>>
	<<output code for up commands>>
	<<output code for down commands>>
}
@ 

The description section is just a documentation chunk,
and hence isn't at all relevant for the C code.

<<output code for description>>=
if (match($line, "description", $indent)) {
	skip_section();
}
@ 

Skipping a section is fairly easy: we just need to check alignments. This is
yet another subroutine that'll come in handy elsewhere.

<<defn2c subroutines>>=
<<[[skip_section]] subroutine>>
@ 

<<[[skip_section]] subroutine>>=
sub skip_section {
	my $struct = $_[0];
	my $indent = ($line =~ /(\s*)[^\s]/) ? $1 : "";

	1 while (nextline && match($line, "", $indent));
}
@ 

The options section is processed to check later that no repetitive
options created by the conversions.

<<output code for options list>>=
if (match($line, "options", $indent)) {
	@options = get_options();
}
print "static option_default _${method}_default[] = {\n";
if (@options) {
	foreach my $o (@options) {
		if ($o =~ m/^\s*(\S*)\s*(.*)\s+--\s+(\S[^[]*)(\s+\[([^]]*)\]\s*)?$/) {
			my $opt = $1;
			my $optargs = $2;
			my $dsc = $3;
			push @variables, $opt;
			if ($4) {
				print "\t{ \"$opt\", \"$5\" },\n";
			}
		}
	}
}
print "\t{ NULL, NULL }\n";
print "};\n";
@ 

Checking the various relevant components of each method is fairly
simple: we need to see if it exists, and if it does, parse and output
it, while if it doesn't, we need to output a place holder.

<<output code for up commands>>=
if (match($line, "up", $indent)) {
	get_commands(${method}, "up");
} else {
	print "static int _${method}_up(interface_defn ifd) { return 0; }\n"
}
@ 

<<output code for down commands>>=
if (match($line, "down", $indent)) {
	get_commands(${method}, "down");
} else {
	print "static int _${method}_down(interface_defn ifd) { return 0; }\n"
}
@

<<output code for conversions>>=
print "static conversion _${method}_conv[] = {\n";
if (match($line, "conversion", $indent)) {
	while (nextline && match($line, "", "$indent  ")) {
		my $foo = $line;
		$foo =~ s/^\s+//;
		$foo =~ m/^\s*(\S+)\s+(\([^)]+\)|\S+)\s*(\S+)?\s*$/;
		my $option = $1;
		my $fn = $2;
		my $newoption = $3;
		if ($fn =~ m/^\((.*)\)$/) {
			my @params = split(/ /, $1);
			$fn = shift(@params);
			foreach (@params) {
				if ($_ =~ m/^"(.*)"$/) {
				    $_ = $1;
				}
			}
			$fn .= (", ".scalar(@params).", (char * []){\"".join("\", \"", @params)."\"}");
		} else {
			$fn .= ", 0, NULL";
		}
		if ($newoption) {
			$newoption =~ s/^=//;
			die "Duplicate option use: $newoption (from $method/$option)" if (grep $_ eq $newoption, @variables);
			push @variables, $newoption;
			print "\t{ \"$option\", \"$newoption\", $fn },\n";
		} else {
			print "\t{ \"$option\", NULL, $fn },\n";
		}
	}
}
print "\t\{ NULL, NULL, NULL, 0, NULL }\n";
print "};\n";
@

<<defn2c subroutines>>=
sub get_commands {
	my $method = $_[0];
	my $mode = $_[1];
	my $function = "_${method}_${mode}";
	my $indent = ($line =~ /(\s*)[^\s]/) ? $1 : "";

	print "static int ${function}(interface_defn *ifd, execfn *exec) {\n";

	while (nextline && match($line, "", $indent)) {
		if ( $match =~ /^(.*[^\s])\s+if\s*\((.*)\)\s*$/ ) {
			print "if ( $2 ) {\n";
			print "  if (!execute(\"$1\", ifd, exec)) return 0;\n";
			print "}\n";
		} elsif ( $match =~ /^(.*[^\s])\s+elsif\s*\((.*)\)\s*$/ ) {
			print "else if ( $2 ) {\n";
			print "  if (!execute(\"$1\", ifd, exec)) return 0;\n";
			print "}\n";
		} elsif ( $match =~ /^(.*[^\s])\s*$/ ) {
			print "{\n";
			print "  if (!execute(\"$1\", ifd, exec)) return 0;\n";
			print "}\n";
		}
	}

	print "return 1;\n";
	print "}\n";
}
@ 

\subsection{Building Manual Pages}

So having C code is all very well, but if you want to ignore all user
problems with a casual ``RTFM!'' there has to be some semblance of an
M for them to R. So we need a script to generate some useful
descriptions of the various methods.

We'll achieve this by making another Perl script, [[defn2man.pl]],
which will generate fragments of [[troff]] that can be catted together
with a general overview of [[ifupdown]] to produce real manpages.

<<implicit rules>>=
%.man: %.defn defn2man.pl
	./defn2man.pl $< > $@
@ 

So we'll use a similar structure to [[defn2c.pl]].

<<defn2man.pl>>=
#!/usr/bin/perl -w

use strict;

<<determine the target architecture>>

# declarations
<<defn2man variables>>

# subroutines
<<defn2man subroutines>>

# main code
<<parse [[.defn]] file and output intermediate structures>>
@ 

As predicted, we'll also incorporate [[nextline]], [[match]] and
[[skip_section]]:

<<defn2man variables>>=
my $line;
my $match;
my $arch = "";
@ 

<<defn2man subroutines>>=
<<[[nextline]] subroutine>>
<<[[our_arch]] function>>
<<[[match]] subroutine>>
<<[[skip_section]] subroutine>>
@ 

We use the same main loop as in [[defn2c]] with absolutely no changes.

The [[get_address_family]] and [[get_architecture]] subroutines are
fairly straight forward:

<<defn2man subroutines>>=
sub get_address_family {
	print ".SH " . uc($match) . " ADDRESS FAMILY\n";
	print "This section documents the methods available in the\n";
	print "$match address family.\n";
	nextline;
}
@ 

<<defn2man subroutines>>=
sub get_architecture {
	$arch = $_[0];
	nextline;
}
@ 

Which only leaves extracting the description and options for each
method. And, of course, this imposes less restrictions of the
[[.defn]] file than [[defn2c.pl]] did. It's a crazy old world.

<<defn2man subroutines>>=
sub get_method {
	my $method = shift;
	my $indent = ($line =~ /(\s*)\S/) ? $1 : "";
	my $description = "";
	my @options = ();

	nextline;
	while ($line and match($line, "", $indent)) {
		if (match($line, "description", $indent)) {
			$description = get_description();
		} elsif (match($line, "options", $indent)) {
			@options = get_options();
		} else {
			skip_section;
		}
	}

	<<output [[$method]] introduction man fragment>>
	<<output [[$description]] man fragment>>
	<<output [[@options]] man fragment>>
}
@

<<output [[$method]] introduction man fragment>>=
print ".SS The $method Method\n";
@ 

Okay. Now our [[$description]] is just the description with any [['\n']]
characters it may've had, but without the leading spaces.

<<defn2man subroutines>>=
sub get_description {
	my $desc = "";
	my $indent = ($line =~ /(\s*)\S/) ? $1 : "";
	while(nextline && match($line, "", $indent)) {
		$desc .= "$match\n";
	}
	return $desc;
}
@ 

We're actually going to be a little tricky here, and allow some formatting
in our descriptions. Basically, we'll allow bold and italic encoding
to be denoted by [[*bold*]] and [[/italics/]] in the wonderful Usenet
tradition. As such, we'll use a cute little function to convert the
Usenet style to \emph{roff}. We'll also take care not to do this conversion
within words (for things like [[/etc/hosts]], eg). Voila:

<<defn2man subroutines>>=
sub usenet2man {
	my $in = shift;
	my $out = "";

	$in =~ s/\s+/ /g;
	while ($in =~ m%^([^*/]*)([*/])([^*/]*)([*/])(.*)$%s) {
		my ($pre, $l, $mid, $r, $post) = ($1, $2, $3, $4, $5);
		if ($l eq $r && " $pre"  =~ m/[[:punct:][:space:]]$/ 
			     && "$post " =~ m/^[[:punct:][:space:]]/) {
			$out .= $pre;
			$out .= ($l eq "*" ? '\fB' : '\fI') . $mid . '\fP';
			($in = $post) =~ s/^\s+/ /;
		} else {
			$out .= $pre . $l;
			$in = $mid . $r . $post;
		}
	} 
	return $out . $in;
}
@

The only further thing to note about this is that we're being careless
and ignoring the possibility of \emph{roff} escape sequences in the input. But
since this is for internal use only, well, too bad. So here we go:

<<output [[$description]] man fragment>>=
if ($description ne "") {
	print usenet2man($description) . "\n";
} else {
	print "(No description)\n";
}
@ 

Damn that was fun.

Reading the options is almost exactly the same as the description,
except we want a list instead of just a string.

<<defn2man subroutines>>=
<<[[get_options]] fragment>>
@ 

<<defn2c subroutines>>=
<<[[get_options]] fragment>>
@ 

<<[[get_options]] fragment>>=
sub get_options {
	my @opts = ();
	my $indent = ($line =~ /(\s*)\S/) ? $1 : "";
	while(nextline && match($line, "", $indent)) {
		push @opts, $match;
	}
	return @opts;
}
@ 

Output is slightly more complicated, but not too much so.

<<output [[@options]] man fragment>>=
print ".PP\n";
print ".B Options\n";
print ".RS\n";
if (@options) {
	foreach my $o (@options) {
		if ($o =~ m/^\s*(\S*)\s*(.*)\s+--\s+(\S[^[]*)(\s+\[([^]]*)\]\s*)?$/) {
			my $opt = $1;
			my $optargs = $2;
			my $dsc = $3;
			$dsc .= (length($5)) ? ". Default value: \"$5\"" : "";
			print ".TP\n";
			print ".BI $opt";
			print " \" $optargs\"" unless($optargs =~ m/^\s*$/);
			print "\n";
			print usenet2man($dsc) . "\n";
		} else {
			print ".TP\n";
			print ".B $o\n";
		}
	}
} else {
	print ".TP\n";
	print "(No options)\n";
}
print ".RE\n";
@ 

\section{Run-time Configuration}

Our module is of the usual form, and we'll make use of a few fairly standard
headers. Please move along, there's nothing to see here.

<<config.c>>=
<<config headers>>
<<config function declarations>>
<<config functions>>
@ 

<<config headers>>=
#include <stdlib.h>
#include <stdio.h>
#include <string.h>
#include <errno.h>
#include <assert.h>
@ 

We'll also make use of some of our other modules. This is, after all,
why we had a single header in the first place.

<<config headers>>=
#include "header.h"
@ 

The key function we're interested in defining here is
[[read_interfaces()]], which will (wait for it) read an interfaces
file. The intention is to make it really easy to deal with the
vagaries of [[/etc/network/interfaces]] anywhere else.

So the first question we need to deal with is ``What's a convenient
form for other functions which deal with interfaces?'' Well, our
answer to that is basically:

\begin{enumerate}
	\item an array of interface names that should be brought up at
	bootup.

	\item a singly linked list to represent the various mappings.

	\item another singly linked list to represent the interface
	definitions themselves.
\end{enumerate}

 These are almost in exact correspondence with the original file.

<<type definitions>>=
typedef struct interfaces_file interfaces_file;
@ 

<<structure definitions>>=
struct interfaces_file {
	allowup_defn *allowups;
	interface_defn *ifaces;
	mapping_defn *mappings;
};
@

So, at run-time, we first need a way of dealing with the [[auto]] and
[[allow-*]] lines. We'll treat [[allow-auto]] and [[auto]] as equivalent,
making that pretty straightforward:

<<type definitions>>=
typedef struct allowup_defn allowup_defn;
@

<<structure definitions>>=
struct allowup_defn {
	allowup_defn *next;

	char *when;
	int max_interfaces;
	int n_interfaces;
	char **interfaces;
};
@ 

We also require a way of representing each interface listed in the
configuration file. This naturally needs to reference an address family
and method, and all the options a user may specify about an interface.

<<type definitions>>=
typedef struct interface_defn interface_defn;
@ 

<<structure definitions>>=
struct interface_defn {
	interface_defn *next;

	char *logical_iface;
	char *real_iface;

	address_family *address_family;
	method *method;

	int automatic;

	int max_options;
	int n_options;
	variable *option;
};
@

The last component in the above, the options, is represented by a
series of name/value pairs, as follows:

<<type definitions>>=
typedef struct variable variable;
@ 

<<structure definitions>>=
struct variable {
	char *name;
	char *value;
};
@ 

We'll define a function to help us build arrays of these [[variables]].
It's fairly straightforward: to add a variable, we simply construct
a new structure and add it at the end of our array of variables,
increasing the size of the array first if necessary. [[set_variable]]
function adds a variable if it doesn't exist yet, or updates one which
is already set but only if the variable name doesn't end with a question
mark. If [[set_variable]] didn't do anything to the variable, it returns
NULL.

<<exported symbols>>=
variable * set_variable(char *filename, char *name, char *value, 
		variable **var, int *n_vars, int *max_vars);
void convert_variables(char *filename, conversion *conversions, 
		interface_defn *ifd);
@

<<config functions>>=
variable * set_variable(char *filename, char *name, char *value, 
		variable **var, int *n_vars, int *max_vars)
{
	/*
	 * if name ends with '?', don't update
	 * the variable if it already exists
	 */
	bool dont_update = false;

	size_t len = strlen(name);

	if (name[len - 1] == '?') {
	    dont_update = true;
	    len--;
	}
	if (strcmp(name, "pre-up") != 0
    	&& strcmp(name, "up") != 0
	    && strcmp(name, "down") != 0
	    && strcmp(name, "post-down") != 0)
	{
		int j;
		for (j = 0; j < *n_vars; j++) {
			if (strncmpz(name, (*var)[j].name, len) == 0)
			{
				if (dont_update) {
					return NULL;
				}

				if ((*var)[j].value == value) {
				    return value;
				}

				free((*var)[j].value);
				(*var)[j].value = strdup(value);
				if (!(*var)[j].value) {
					<<report internal error and die>>
				}

				return &((*var)[j]);
			}
		}
	}

	if (*n_vars >= *max_vars) {
		variable *new_var;
		*max_vars += 10;
		new_var = realloc(*var, sizeof(variable) * *max_vars);
		if (new_var == NULL) {
			<<report internal error and die>>
		}
		*var = new_var;
	}

	(*var)[*n_vars].name = strndup(name, len);
	(*var)[*n_vars].value = strdup(value);

	if (!(*var)[*n_vars].name) {
		<<report internal error and die>>
	}

	if (!(*var)[*n_vars].value) {
		<<report internal error and die>>
	}

	(*n_vars)++;
	return &((*var)[(*n_vars) - 1]);
}

void convert_variables(char *filename, conversion *conversions, 
		interface_defn *ifd)
{
	conversion *c;
	for (c = conversions; c && c->option && c->fn; c++) {
		if (strcmp(c->option, "iface") == 0) {
			if (c->newoption) {
				variable *o = set_variable(filename, c->newoption, ifd->real_iface,
					&ifd->option, &ifd->n_options, &ifd->max_options);
				if (o)
					c->fn(ifd, &o->value, c->argc, c->argv);
				continue;
			}
		}

		int j;
		for (j = 0; j < ifd->n_options; j++) {
			if (strcmp(ifd->option[j].name, c->option) == 0)
			{
				if (c->newoption) {
					variable *o = set_variable(filename, c->newoption, ifd->option[j].value,
						&ifd->option, &ifd->n_options, &ifd->max_options);
					if (o)
						c->fn(ifd, &o->value, c->argc, c->argv);
				} else {
					variable *o = &(ifd->option[j]);
					c->fn(ifd, &o->value, c->argc, c->argv);
				}
			}
		}
	}
}
@

In addition, we want to represent each mapping in the configuration
file. This is somewhat simpler, since each mapping is entirely self
contained, and doesn't need to reference previously determined address
families or methods or anything.

<<type definitions>>=
typedef struct mapping_defn mapping_defn;
@ 

<<structure definitions>>=
struct mapping_defn {
	mapping_defn *next;

	int max_matches;
	int n_matches;
	char **match;

	char *script;

	int max_mappings;
	int n_mappings;
	char **mapping;
};
@ 

We can thus begin to instantiate our actual function. What we want is
something that, given the name of a file, will produce the appropriate
linked list of interfaces defined in it, or possibly give some sort of
helpful error message. Pretty simple, hey?

<<exported symbols>>=
interfaces_file *read_interfaces(char *filename);
interfaces_file *read_interfaces_defn(interfaces_file *defn, char *filename);
@ 

<<config functions>>=
interfaces_file *read_interfaces(char *filename) {
	interfaces_file *defn;

	<<allocate defn or [[return NULL]]>>
	return read_interfaces_defn(defn, filename);
}

interfaces_file *read_interfaces_defn(interfaces_file *defn, char *filename) {
	<<variables local to read interfaces>>

	<<open file or [[return NULL]]>>

	while (<<we've gotten a line from the file>>) {
		<<process the line>>
	}
	if (<<an error occurred getting the line>>) {
		<<report internal error and die>>
	}

	<<close file>>

	return defn;
}
@ 

<<allocate defn or [[return NULL]]>>=
defn = malloc(sizeof(interfaces_file));
if (defn == NULL) {
	return NULL;
}
defn->allowups = NULL;
defn->mappings = NULL;
defn->ifaces = NULL;

if (!no_loopback) {
	interface_defn * lo_if = malloc(sizeof(interface_defn));
	if (!lo_if) {
		<<report internal error and die>>
	}

	*lo_if = (interface_defn){
	    .logical_iface = strdup(LO_IFACE),
	    .max_options = 0,
	    .address_family = &addr_inet,
	    .method = get_method(&addr_inet, "loopback"),
	    .n_options = 0,
	    .option = NULL,
	    .next = NULL
	};

	defn->ifaces = lo_if;

	add_allow_up(__FILE__, __LINE__, get_allowup(&defn->allowups, "auto"), lo_if->logical_iface);
}
@ 

\subsection{File Handling}

So, the first and most obvious thing to deal with is the file
handling. Nothing particularly imaginative here.

<<variables local to read interfaces>>=
FILE *f;
int line;
@ 

<<open file or [[return NULL]]>>=
f = fopen(filename, "r");
if ( f == NULL ) return NULL;
line = 0;
@

<<close file>>=
fclose(f);
line = -1;
@

\subsection{Line Parsing}

Our next problem is to work out how to read a single line from our
input file. While this is nominally easy, we also want to deal nicely
with things like continued lines, comments, and very long lines.

So we're going to have to write and make use of a complicated little
function, which we'll imaginatively call [[get_line()]]. It will need
a pointer to the file it's reading from, as well as a buffer to store
the line it finds. Since this buffer's size can't be known in advance
we'll need to make it [[realloc()]]-able, which means we need to pass
around references to both the buffer's location (which may change),
and it's size (which probably will). Our function declaration is thus:

<<config function declarations>>=
static int get_line(char **result, size_t *result_len, FILE *f, int *line);
@ 

To use it, we'll need a couple of variables to stores the buffer's
location, and it's current length.

<<variables local to read interfaces>>=
char *buf = NULL;
size_t buf_len = 0;
@ 

Given these, and presuming we can actually implement the function, our
key chunk can thus be implemented simply as:

<<we've gotten a line from the file>>=
get_line(&buf,&buf_len,f,&line)
@ 

We'll also add the requirement that errors are indicated by the
[[errno]] variable being non-zero, which is usual and reasonable for
all the circumstances where [[get_line()]] might have problems.

<<config headers>>=
#include <errno.h>
@ 

<<an error occurred getting the line>>=
ferror(f) != 0
@ 

Actually defining the function is, as you'd probably imagine, a little
more complicated. We begin by reading a line from the file. If it was
a comment (that is, it has a [[#]] character at the first non-blank
position) then we try again. Otherwise, if the line is continued
(indicated by a [[\]] character at the very end of the line) we append
the next line to the buffer. We go to a little bit of effort to trim
whitespace, and finally return a boolean result indicating whether we
got a line or not.

<<config functions>>=
static int get_line(char **result, size_t *result_len, FILE *f, int *line) {
	<<variables local to get line>>

	do {
		<<clear buffer>>
		<<append next line to buffer, or [[return 0]]>>
		<<trim leading whitespace>>
	} while (<<line is a comment>>);

	while (<<buffer is continued>>) {
		<<remove continuation mark>>
		<<append next line to buffer, or [[return 0]]>>
	}

	<<trim trailing whitespace>>

	return 1;
}
@ 

In order to do string concatenation efficiently, we'll keep track of
where the end of the line so far is --- this is thus where the
terminating [[NUL]] will be by the end of the function.

<<variables local to get line>>=
size_t pos;
@ 

We can thus clear the buffer by simply resetting where we append new
text to the beginning of the buffer. What could be simpler?

<<clear buffer>>=
pos = 0;
@ 

We'll be making use of the [[fgets()]] function to read the line
(rather than, say, [[fgetc()]]) so to get an entire line we may have
to make multiple attempts (if the line is bigger than our
buffer). Realising this, and the fact that we may not have any
allocated space for our buffer initially, we need a loop something
like:

<<append next line to buffer, or [[return 0]]>>=
do {
	<<reallocate buffer as necessary, or [[return 0]]>>
	<<get some more of the line, or [[return 0]]>>
} while(<<the line isn't complete>>);

<<remove trailing newline>>

(*line)++;

assert( (*result)[pos] == '\0' );
@ 

When reallocating the buffer, we need to make sure it increases in
chunks large enough that we don't have to do this too often, but not
so huge that we run out of memory just to read an 81 character line.
We'll use two fairly simple heuristics for this: if we've got room to
add no more than 10 characters, we may as well reallocate the buffer,
and when reallocating, we want to more or less double the buffer, but
we want to at least add 80 characters. So we do both.

<<reallocate buffer as necessary, or [[return 0]]>>=
if (*result_len - pos < 10) {
	char *newstr = realloc(*result, *result_len * 2 + 80);
	if (newstr == NULL) {
		return 0;
	}
	*result = newstr;
	*result_len = *result_len * 2 + 80;
}
@ 

The only time we need to keep reading is when the buffer wasn't big
enough for the entire line. This is indicated by a full buffer, with
no newline at the end. There is, actually, one case where this can
happen legitimately --- where the last line of the file is
\emph{exactly} the length of the buffer. We need to detect this
because [[fgets()]] will return [[NULL]] and indicate that it's hit
the end of the file, but we won't want to indicate that until the
\emph{next} time we try to get a line. Complicated, isn't it?

<<the line isn't complete>>=
pos == *result_len - 1 && (*result)[pos-1] != '\n'
@ 

So having thought through all that, actually working with [[fgets()]]
is fairly simple, especially since we deal with the actual errors
elsewhere. All we need to do is make the call, update [[pos]] and
check that the problems [[fgets()]] may have actually bother us.

<<get some more of the line, or [[return 0]]>>=
if (!fgets(*result + pos, *result_len - pos, f)) {
	if (ferror(f) == 0 && pos == 0) return 0;
	if (ferror(f) != 0) return 0;
}
pos += strlen(*result + pos);
@ 

[[fgets()]] leaves a [[\n]] in our buffer in some cases. We're never
actually interested in it, however, so it's a good move to get rid of
it.

<<remove trailing newline>>=
if (pos != 0 && (*result)[pos-1] == '\n') {
	(*result)[--pos] = '\0';
}
@ 


Pretty simple, hey? Now the next thing we want to do is get rid of
some of the whitespace lying about. This is all pretty basic, and just
involves finding where the whitespace begins and ends, and, well,
getting rid of it.

<<config headers>>=
#include <ctype.h>
@ 

<<trim leading whitespace>>=
{ 
	int first = 0; 
	while (isspace((*result)[first]) && (*result)[first]) {
		first++;
	}

	memmove(*result, *result + first, pos - first + 1);
	pos -= first;
}
@ 

<<trim trailing whitespace>>=
while (isspace((*result)[pos-1])) { /* remove trailing whitespace */
	pos--;
}
(*result)[pos] = '\0';
@ 

As we mentioned earlier, a line is a comment iff it's first character
is a [[#]] symbol. Similarly, it's continued iff it's very last
character is a [[\]]. And, rather obviously, if we want to remove a
single trailing [[\]], we can do so by changing it to a [[NUL]].

<<line is a comment>>=
(*result)[0] == '#'
@ 

<<buffer is continued>>=
(*result)[pos-1] == '\\'
@ 

<<remove continuation mark>>=
(*result)[--pos] = '\0';
@ 

\subsection{Line Processing}

So. We've gone to a lot of trouble to get a line that we can parse
with a snap of our fingers, so we probably better jump to it, to mix
some \emph{clich\'e's}.

We have two alternative bits of state to maintain between lines: either
what interface we're currently defining, or what mapping we're currently
defining.

<<variables local to read interfaces>>=
interface_defn *currif = NULL;
mapping_defn *currmap = NULL;
enum { NONE, IFACE, MAPPING } currently_processing = NONE;
@ 

Since our configuration files are pretty basic, we can work out what
any particular line means based on the first word in it. To cope with
this, we'll thus make use of a couple of variables, one to store the
first word, and the other to store the rest of the line.

<<variables local to read interfaces>>=
char firstword[80];
char *rest;
@ 

To initialize these variables we'll make use of a function I'm overly
fond of called [[next_word()]]. It copies the first word in a string
to a given buffer, and returns a pointer to the rest of the buffer.

<<config function declarations>>=
static char *next_word(char *buf, char *word, int maxlen);
@

<<config functions>>=
static char *next_word(char *buf, char *word, int maxlen) {
	if (!buf) return NULL;
	if (!*buf) return NULL;

	while(!isspace(*buf) && *buf) {
		if (maxlen-- > 1) *word++ = *buf;
		buf++;
	}
	if (maxlen > 0) *word = '\0';

	while(isspace(*buf) && *buf) buf++;

	return buf;
}
@ 

So after all this, there are basically three different sorts of line
we can get: the start of a new interface, the start of a new mapping,
or an option for whatever interface we're currently working with.
Note that we check for blank lines, but \emph{not} for options with
empty values.  This has to be done on a case-by-case basis.

<<process the line>>=
rest = next_word(buf, firstword, 80);
if (rest == NULL) continue; /* blank line */

if (strcmp(firstword, "mapping") == 0) {
	<<process [[mapping]] line>>
	currently_processing = MAPPING;
} else if (strcmp(firstword, "source") == 0) {
	<<process [[source]] line>>
	currently_processing = NONE;
} else if (strcmp(firstword, "iface") == 0) {
	<<process [[iface]] line>>
	currently_processing = IFACE;
} else if (strcmp(firstword, "auto") == 0) {
	<<process [[auto]] line>>
	currently_processing = NONE;
} else if (strncmp(firstword, "allow-", 6) == 0 && strlen(firstword) > 6) {
	<<process [[allow-]] line>>
	currently_processing = NONE;
} else {
	<<process option line>>
}
@ 

<<process option line>>=
switch(currently_processing) {
	case IFACE:
		<<process iface option line>>
		break;
	case MAPPING:
		<<process mapping option line>>
		break;
	case NONE:
	default:
		<<report bad option and die>>
}
@ 

\subsubsection{Source Line}

When processing the [[source]] stanza, we use [[wordexp]] function to expand wildcards 
and environment variables. 

<<config headers>>=
#include <wordexp.h>
@

We use [[WRDE_NOCMD]] flag, so no command substitution occurs
because of security concerns. Then we go through the output array and read interfaces 
recursively into already allocated [[defn]].

<<process [[source]] line>>=
wordexp_t p;
char ** w;
size_t i;
int fail = wordexp(rest, &p, WRDE_NOCMD);
if (!fail)
{
	w = p.we_wordv;
	for (i = 0; i < p.we_wordc; i++)
	{
		read_interfaces_defn(defn, w[i]);
	}
	wordfree(&p);
}
@

\subsubsection{Mapping Line}

Declaring a new mapping is reasonably copewithable --- we need to process
a few things, but they're reasonably easy to handle.

The main weirdness is that we're processing the [[mapping]] line itself
and the rest of the stanza in separate blocks of code. So this first
chunk just needs to do the basics of initialising the data structure,
but can't really fill in all that much of it.

<<process [[mapping]] line>>=
<<allocate new mapping>>
<<parse mapping interfaces>>
<<set other mapping options to defaults>>
<<add to list of mappings>>
@ 

<<allocate new mapping>>=
currmap = malloc(sizeof(mapping_defn));
if (currmap == NULL) {
	<<report internal error and die>>
}
@ 

<<parse mapping interfaces>>=
currmap->max_matches = 0;
currmap->n_matches = 0;
currmap->match = NULL;

while((rest = next_word(rest, firstword, 80))) {
	if (currmap->max_matches == currmap->n_matches) {
		char **tmp;
		currmap->max_matches = currmap->max_matches * 2 + 1;
		tmp = realloc(currmap->match, 
			sizeof(*tmp) * currmap->max_matches);
		if (tmp == NULL) {
			currmap->max_matches = (currmap->max_matches - 1) / 2;
			<<report internal error and die>>
		}
		currmap->match = tmp;
	}

	currmap->match[currmap->n_matches++] = strdup(firstword);
}
@ 

<<set other mapping options to defaults>>=
currmap->script = NULL;

currmap->max_mappings = 0;
currmap->n_mappings = 0;
currmap->mapping = NULL;
@ 

<<add to list of mappings>>=
{
	mapping_defn **where = &defn->mappings;
	while(*where != NULL) {
		where = &(*where)->next;
	}
	*where = currmap;
	currmap->next = NULL;
}
@ 

So that's that. But as mentioned, we also need to cope with the options
within the stanza, as well as the lead in. As before, it's not really
complicated, and we do it thusly:

<<process mapping option line>>=
if (strcmp(firstword, "script") == 0) {
	<<handle [[script]] line>>
} else if (strcmp(firstword, "map") == 0) {
	<<handle [[map]] line>>
} else {
	<<report bad option and die>>
}
@ 

<<handle [[script]] line>>=
if (currmap->script != NULL) {
	<<report duplicate script in mapping and die>>
} else {
	currmap->script = strdup(rest);
}
@ 

<<handle [[map]] line>>=
if (currmap->max_mappings == currmap->n_mappings) {
	char **opt;
	currmap->max_mappings = currmap->max_mappings * 2 + 1;
	opt = realloc(currmap->mapping, sizeof(*opt) * currmap->max_mappings);
	if (opt == NULL) {
		<<report internal error and die>>
	}
	currmap->mapping = opt;
}
currmap->mapping[currmap->n_mappings] = strdup(rest);
currmap->n_mappings++;
@ 

\subsubsection{Interface line}

Declaring a new interface follows the same pattern, but is somewhat more
interesting and some more complicated data structures are involved.

<<process [[iface]] line>>=
{
	<<variables local to process [[iface]] line>>

	<<allocate new interface>>

	<<parse interface settings>>

	<<set iface name>>
	<<set address family>>
	<<set method>>
	<<set other interface options to defaults>>

	<<add to list of interfaces>>
}
@ 

We'll deal with each of these phases one by one and pretty much in
order, so prepare yourself for the intense excitement of memory
allocation!

<<allocate new interface>>=
currif = malloc(sizeof(interface_defn));
if (!currif) {
	<<report internal error and die>>
}
@ 

When we introduce a new interface, we simultaneously name the
interface, the address family, and the method. We cope with this by,
well, getting somewhere to store each of them, and then, well, storing
them.

<<variables local to process [[iface]] line>>=
char iface_name[80];
char address_family_name[80];
char method_name[80];
@ 

<<parse interface settings>>=
rest = next_word(rest, iface_name, 80);
rest = next_word(rest, address_family_name, 80);
rest = next_word(rest, method_name, 80);

if (rest == NULL) {
	<<report too few parameters for iface line and die>>
}

if (rest[0] != '\0') {
	<<report too many parameters for iface line and die>>
}
@ 

We then want to store the interface name.

<<set iface name>>=
currif->logical_iface = strdup(iface_name);
if (!currif->logical_iface) {
	<<report internal error and die>>
}
@ 

Setting the address family is a little more involved, because it's not
very useful to know what the name of the address family is, you really
want to know all the details recorded in the appropriate
[[address_family]] structure. So we'll make use of a little helper
function, called [[get_address_family()]] to convert the useless
string, to the hopefully less useless structure.

<<config function declarations>>=
static address_family *get_address_family(address_family *af[], char *name);
@ 

<<set address family>>=
currif->address_family = get_address_family(addr_fams, address_family_name);
if (!currif->address_family) {
	<<report unknown address family and die>>
}
@ 

Of course, we probably need to actually implement the function too. We
won't do anything particularly fancy here, just a simple linear
search. \emph{Should this really be here, or an exported symbol from
[[addrfam.c]]? --- aj}

<<config functions>>=
static address_family *get_address_family(address_family *af[], char *name) {
	int i;
	for (i = 0; af[i]; i++) {
		if (strcmp(af[i]->name, name) == 0) {
			return af[i];
		}
	}
	return NULL;
}
@

We do something incredibly similar when dealing with the method the
user wishes to use, and we do it for incredibly similar reasons. Again
we declare a cute little helper function, this time imaginatively
called [[get_method()]], and then go and use it and implement in
almost exactly the same way as before. I told you this was going to be
a thrill. \emph{The same note applies here, too --- aj}

<<config function declarations>>=
static method *get_method(address_family *af, char *name);
@ 

<<set method>>=
currif->method = get_method(currif->address_family, method_name);
if (!currif->method) {
	<<report unknown method and die>>
	return NULL; /* FIXME */
}
@

<<config functions>>=
static method *get_method(address_family *af, char *name) {
	int i;
	for (i = 0; i < af->n_methods; i++) {
		if (strcmp(af->method[i].name, name) == 0) {
			return &af->method[i];
		}
	}
	return NULL;
}
@

You'll continue to be enthralled as we set the remaining options to
some default values.

<<set other interface options to defaults>>=
currif->automatic = 1;
currif->max_options = 0;
currif->n_options = 0;
currif->option = NULL;
@ 

Since we want to keep the interfaces in order, we have to go all the
way to the end of the list of interfaces to add the new interface, and
we can hence set the [[next]] pointer to NULL in all cases. Gee. Whiz.

We allow multiple interface definitions just to cope well with multiple
network addresses which can be assigned to the same interface (which is
a standard feature in IPv6, for example).

<<add to list of interfaces>>=
{
	interface_defn **where = &defn->ifaces; 
	while(*where != NULL) {
		where = &(*where)->next;
	}

	*where = currif;
	currif->next = NULL;
}
@ 

Dealing with the per-interface options is the next thing to deal
with. 

<<process iface option line>>=
<<convert [[post-up]] and [[pre-down]] aliases to [[up]] and [[down]]>>
<<check for duplicate options>>
<<add option>>
@

<<convert [[post-up]] and [[pre-down]] aliases to [[up]] and [[down]]>>=
if (strcmp(firstword, "post-up") == 0) {
	strcpy(firstword, "up");
}
if (strcmp(firstword, "pre-down") == 0) {
	strcpy(firstword, "down");
} 
@

<<check for duplicate options>>=
{
	int i;

	if (strlen (rest) == 0) {
		<<report empty option and die>>
	}

	if (strcmp(firstword, "pre-up") != 0 
	    && strcmp(firstword, "up") != 0
	    && strcmp(firstword, "down") != 0
	    && strcmp(firstword, "post-down") != 0)
        {
		for (i = 0; i < currif->n_options; i++) {
			if (strcmp(currif->option[i].name, firstword) == 0) {
				size_t l = strlen(currif->option[i].value);
				currif->option[i].value = realloc(currif->option[i].value, l + strlen(rest) + 2); /* 2 for NL and NULL */
				if (!currif->option[i].value) {
					<<report internal error and die>>
				}

				currif->option[i].value[l] = '\n';
				strcpy(&(currif->option[i].value[l + 1]), rest);
				rest = currif->option[i].value;
			}
		}
	}
}
@ 

Given the previous definition of [[set_variable()]] adding an option
is straightforward.

<<add option>>=
set_variable(filename, firstword, rest,
	 &currif->option, &currif->n_options, &currif->max_options);
@

\subsubsection{Auto and Allow Lines}

Processing the [[auto]] and [[allow-]] lines is pretty straightforward
after the above, we just need to add each parameter to the list and
check for duplicates. Since we're doing essentially the same thing twice,
we'll break the common part out into a function.

<<process [[auto]] line>>=
allowup_defn *auto_ups = get_allowup(&defn->allowups, "auto");
if (!auto_ups) {
	<<report internal error and die>>
}
while((rest = next_word(rest, firstword, 80))) {
	if (!add_allow_up(filename, line, auto_ups, firstword))
		return NULL;
}
@ 
<<process [[allow-]] line>>=
allowup_defn *allow_ups = get_allowup(&defn->allowups, firstword + 6);
if (!allow_ups) {
	<<report internal error and die>>
}
while((rest = next_word(rest, firstword, 80))) {
	if (!add_allow_up(filename, line, allow_ups, firstword))
		return NULL;
}
@ 

<<config function declarations>>=
allowup_defn *get_allowup(allowup_defn **allowups, char *name);

<<config functions>>=
allowup_defn *get_allowup(allowup_defn **allowups, char *name) {
	for (; *allowups; allowups = &(*allowups)->next) {
		if (strcmp((*allowups)->when, name) == 0) break;
	}
	if (*allowups == NULL) {
		*allowups = malloc(sizeof(allowup_defn));
		if (*allowups == NULL) return NULL;
		(*allowups)->when = strdup(name);
		(*allowups)->next = NULL;
		(*allowups)->max_interfaces = 0;
		(*allowups)->n_interfaces = 0;
		(*allowups)->interfaces = NULL;
	}
	return *allowups;
}
@

We'll want to export a little helper function to make finding the appropriate
allowup easier too:

<<exported symbols>>=
allowup_defn *find_allowup(interfaces_file *defn, char *name);
@

<<config functions>>=
allowup_defn *find_allowup(interfaces_file *defn, char *name) {
	allowup_defn *allowups = defn->allowups;
	for (; allowups; allowups = allowups->next) {
		if (strcmp(allowups->when, name) == 0) break;
	}
	return allowups;
}
@

<<config function declarations>>=
allowup_defn *add_allow_up(char *filename, int line,
	 allowup_defn *allow_up, char *iface_name);
@

<<config functions>>=
allowup_defn *add_allow_up(char *filename, int line,
	allowup_defn *allow_up, char *iface_name)
{
	<<check [[iface_name]] isn't already an [[allow_up]] interface>>
	<<add [[iface_name]] as an [[allow_up]] interface or die>>
	return allow_up;
}
@

<<check [[iface_name]] isn't already an [[allow_up]] interface>>=
{
	int i;

	for (i = 0; i < allow_up->n_interfaces; i++) {
		if (strcmp(iface_name, allow_up->interfaces[i]) == 0) {
			return allow_up;
		}
	}
}
@

<<add [[iface_name]] as an [[allow_up]] interface or die>>=
if (allow_up->n_interfaces == allow_up->max_interfaces) {
	char **tmp;
	allow_up->max_interfaces *= 2;
	allow_up->max_interfaces++;
	tmp = realloc(allow_up->interfaces, 
		sizeof(*tmp) * allow_up->max_interfaces);
	if (tmp == NULL) {
		<<report internal error and die>>
	}
	allow_up->interfaces = tmp;
}

allow_up->interfaces[allow_up->n_interfaces] = strdup(iface_name);
allow_up->n_interfaces++;
@

\subsection{Error Handling}

We don't do anything too fancy about handling errors that occur, we
just print out a hopefully helpful error message, and return from the
function. \emph{We probably should also go to some effort to close files,
and free memory, but well, you know. Maybe version $n+1$. --- aj}

<<report internal error and die>>=
perror(filename);
return NULL;
@ 

<<report too few parameters for iface line and die>>=
fprintf(stderr, "%s:%d: too few parameters for iface line\n", filename, line);
return NULL;
@

<<report too many parameters for iface line and die>>=
fprintf(stderr, "%s:%d: too many parameters for iface line\n", filename, line);
return NULL;
@

<<report unknown address family and die>>=
fprintf(stderr, "%s:%d: unknown address type\n", filename, line);
return NULL;
@

<<report unknown method and die>>=
fprintf(stderr, "%s:%d: unknown method\n", filename, line);
return NULL;
@

<<report unknown interface and die>>=
fprintf(stderr, "Unknown interface %s\n", iface);
return 1;
@

<<report [[iface_name]] as [[allow_up]] duplicate, die>>=
fprintf(stderr, "%s:%d: interface %s declared allow-%s twice\n", 
	filename, line, iface_name, allow_up->when);
return NULL;
@

<<report duplicate script in mapping and die>>=
fprintf(stderr, "%s:%d: duplicate script in mapping\n", filename, line);
return NULL;
@ 

<<report bad option and die>>=
fprintf(stderr, "%s:%d: misplaced option\n", filename, line);
return NULL;
@

<<report empty option and die>>=
fprintf(stderr, "%s:%d: option with empty value\n", filename, line);
return NULL;
@ 

\section{Execution}

The [[execute]] module will be laid out in the standard manner, and
will make use of the usual header files.

<<execute.c>>=
<<execute headers>>
<<execute global variables>>
<<execute function declarations>>
<<execute functions>>
@ 

<<execute headers>>=
#include <stdio.h>
#include <ctype.h>
#include <stdlib.h>
#include <string.h>
#include <assert.h>

#include "header.h"
@

The key functions we export from here are all the functions that as a
fairly direct result run some executable.

\begin{itemize}
	\item [[iface_up()]] and [[iface_down()]] which will actually
	configure or deconfigure an interface.

	\item [[iface_list()]] which permits querying a list of known
	interfaces by class.

	\item [[iface_query()]], to query configuration details about a
	specific interface by name.

	\item [[execute()]] which will take an interface definition and
	a command and fill in the details from the first into the
	second, and the execute the result. This is basically just a
	callback for the address family module.

	\item [[run_mapping()]] which will run a mapping script and
	determine if a new logical interface should be selected.
\end{itemize}

We'll discuss each of these in order.

\subsection{Interface Configuration and Deconfiguration}

Most of the complexity is involved in implementing the [[iface_up()]] and
[[iface_down()]] functions. These are complicated enough that an explanatory
diagram is probably useful:

\begin{center}
\includegraphics[height=60mm]{execution}
\end{center}

At a conceptual level, [[iface_up()]] and [[iface_down()]] have a
reasonably straightforward job: they have to run one set of scripts,
then configure or deconfigure the interface, then run another set of
scripts.

This is complicated slightly in that they also have to handle the
possibility that some of an interface required arguments may be missing
(in which case none of the commands should be attempted), and that some
of the commands may fail (in which case none of the following commands
should be attempted). We've already encoded most of the early-abort
logic for the latter case into the address family definitions; so the way
we'll handle the the former case is simply to call the address family's
method [[up()]] or [[down()]] twice: once to ensure all the variables are
appropriately filled out, and once to actually configure the interface.

\subsubsection{Command checking}

As such, we'll make use of two execution functions, each of which take
one parameter, a shell command. We'll uninventively call these [[doit()]]
and [[check()]]. They'll return 0 on failure, non-zero on success.

[[check()]] is thus fairly trivial:

<<execute function declarations>>=
static int check(char *str);
@

<<execute functions>>=
static int check(char *str) {
	return str != NULL;
}
@ 

\subsubsection{Environment handling}

[[doit()]] is much more complicated, mainly by the fact that we
don't simply want to just run the programs, but because we also want
to setup a sanitized environment. In particular, we want to make the
environment variables [[IFACE]], and [[MODE]] available (eg, [[eth0]] and
[[start]] respectively), and we want to export all the given options as
[[IF_OPTION]], with some sanitisation.

We'll do this just once per interface rather than once per command,
and so we'll use a global variable to store our new environment, and a
special function which will initialise it for us.

<<execute global variables>>=
static char **environ = NULL;
@ 

[[environ]] will be in the format used by the [[execle()]] function call,
that is, a [[NULL]]-terminated array of strings of the form [[foo=bar]].

<<execute function declarations>>=
static void set_environ(interface_defn *iface, char *mode, char *phase);
@

Our function then will be:

<<execute functions>>=
static void set_environ(interface_defn *iface, char *mode, char *phase) {
	<<variables local to set environ>>
	int i;
	const int n_env_entries = iface->n_options + 8;

	<<initialise environ [[n_env_entries]]>>

	for (i = 0; i < iface->n_options; i++) {
		<<[[continue]] if option is a command>>

		<<add [[IF_]]option to environment>>
	}

	<<add [[IFACE]] to environment>>
	<<add [[LOGICAL]] to environment>>
	<<add [[ADDRFAM]] to environment>>
	<<add [[METHOD]] to environment>>

	<<add [[MODE]] to environment>>
	<<add [[PHASE]] to environment>>
	<<add [[VERBOSITY]] to environment>>

	<<add [[PATH]] to environment>>
}
@

Since we keep adding at the end, we'll make use of a pointer to keep track
of where the end actually is, namely:

<<variables local to set environ>>=
char **environend;
@

Initialising thus becomes:

<<initialise environ [[n_env_entries]]>>=
<<clear environ if necessary>>
environ = malloc(sizeof(char*) * (n_env_entries + 1 /* for final NULL */));
environend = environ; 
*environend = NULL;
@

<<clear environ if necessary>>=
{
	char **ppch;
	if (environ != NULL) {
		for (ppch = environ; *ppch; ppch++) {
			free(*ppch);
			*ppch = NULL;
		}
		free(environ);
		environ = NULL;
	}
}
@

Our continue chunk is also fairly straight forward:

<<[[continue]] if option is a command>>=
if (strcmp(iface->option[i].name, "pre-up") == 0
    || strcmp(iface->option[i].name, "up") == 0
    || strcmp(iface->option[i].name, "down") == 0
    || strcmp(iface->option[i].name, "post-down") == 0)
{
	continue;
}
@

We'll make use of a small helper function for actually setting the
environment. This function will handle [[malloc]]ing enough memory, and
ensuring the environment variable name is reasonably sensible. It'll
take three parameters: a [[printf]]-style format string presumed to
contain two [[%s]]s, and the two parameters to that format string.

<<execute function declarations>>=
static char *setlocalenv(char *format, char *name, char *value);
@

We can then go ahead and fill in the environment.

<<add [[IF_]]option to environment>>=
*(environend++) = setlocalenv("IF_%s=%s", iface->option[i].name,
                              iface->option[i].value ? iface->option[i].value : "");
*environend = NULL;
@

<<add [[IFACE]] to environment>>=
*(environend++) = setlocalenv("%s=%s", "IFACE", iface->real_iface);
*environend = NULL;
@

<<add [[LOGICAL]] to environment>>=
*(environend++) = setlocalenv("%s=%s", "LOGICAL", iface->logical_iface);
*environend = NULL;
@

<<add [[MODE]] to environment>>=
*(environend++) = setlocalenv("%s=%s", "MODE", mode);
*environend = NULL;
@

<<add [[PHASE]] to environment>>=
*(environend++) = setlocalenv("%s=%s", "PHASE", phase); 
*environend = NULL;
@

<<add [[PATH]] to environment>>=
*(environend++) = setlocalenv("%s=%s", "PATH", "/usr/local/sbin:/usr/local/bin:/usr/sbin:/usr/bin:/sbin:/bin");
*environend = NULL;
@

<<add [[VERBOSITY]] to environment>>=
*(environend++) = setlocalenv("%s=%s", "VERBOSITY", verbose ? "1" : "0");
*environend = NULL;
@

<<add [[ADDRFAM]] to environment>>=
*(environend++) = setlocalenv("%s=%s", "ADDRFAM", iface->address_family->name);
*environend = NULL;
@

<<add [[METHOD]] to environment>>=
*(environend++) = setlocalenv("%s=%s", "METHOD", iface->method->name);
*environend = NULL;
@

Our helper function then will then be something like:

<<execute functions>>=
static char *setlocalenv(char *format, char *name, char *value) {
	char *result;

	<<allocate memory for [[result]]>>

	sprintf(result, format, name, value);

	<<tidy [[result]]>>

	return result;
}
@

Allocating the memory is fairly straightforward (although working out
exactly how much memory involves a little guesswork, and assuming the
caller passes in a reasonable [[format]]).

<<allocate memory for [[result]]>>=
result = malloc(strlen(format)   /* -4 for the two %s's */
                + strlen(name) 
                + strlen(value) 
                + 1);
if (!result) {
	perror("malloc");
	exit(1);
}
@

And finally, tidying the result is a fairly simple matter of eliding all
the characters we don't like, or translating them to ones we do like. We
do like upper case letters, digits and underscores; and we're willing
to translate hyphens and lower case letters. So here we go.

<<tidy [[result]]>>=
{
	char *here, *there;

	for(here = there = result; *there != '=' && *there; there++) {
		if (*there == '-') *there = '_';
		if (isalpha(*there)) *there = toupper(*there);

		if (isalnum(*there) || *there == '_') {
			*here = *there;
			here++;
		}
	}
	memmove(here, there, strlen(there) + 1);
}
@

\subsubsection{Command Execution}

Our [[doit()]] function is then essentially a rewrite of the standard
[[system()]] function call. The only additions are that we setup our
child's environment as discussed previously, and we make use of two
external globals, [[no_act]] and [[verbose]] and modify our behaviour
based on those.

<<exported symbols>>=
int doit(char *str);
@

<<execute functions>>=
int doit(char *str) {
	assert(str);
	bool ignore_status = false;
	if (*str == '-') {
	    ignore_status = true;
	    str++;
	}

	if (verbose || no_act) {
		fprintf(stderr, "%s\n", str);
	}
	if (!no_act) {
		pid_t child;
		int status;

		fflush(NULL);
		setpgid(0, 0);
		switch(child = fork()) {
		    case -1: /* failure */
			return 0;
		    case 0: /* child */
			execle("/bin/sh", "/bin/sh", "-c", str, NULL, environ);
			exit(127);
		    default: /* parent */
		    	break;
		}
		waitpid(child, &status, 0);
		if (ignore_status)
			return 1;

		if (!WIFEXITED(status) || WEXITSTATUS(status) != 0)
			return 0;
	}
	return 1;
}
@

\subsubsection{Executing a list of commands}

In addition to the above, we also need a function to cope with running
all the [[pre-up]] commands and so forth.

<<exported symbols>>=
int execute_options(interface_defn *ifd, execfn *exec, char *opt);
int execute_scripts(interface_defn *ifd, execfn *exec, char *opt);
@ 

All we need to do for this is to iterate through the options in the
interface definition, and execute whichever ones are the right type,
and call the [[run-parts]] command on the appropriate directory of
scripts. That doesn't make for thrilling code.

This function will generally have [[doit]] passed in as the [[exec]]
parameter.

<<execute functions>>=
int execute_options(interface_defn *ifd, execfn *exec, char *opt) {
	int i;
	for (i = 0; i < ifd->n_options; i++) {
		if (strcmp(ifd->option[i].name, opt) == 0) {
			if (!(*exec)(ifd->option[i].value)) {
				return 0;
			}
		}
	}
	return 1;
}

int execute_scripts(interface_defn *ifd, execfn *exec, char *opt) {
	if (!run_scripts) return 1;

	char buf[100];
	snprintf(buf, sizeof(buf), "run-parts %s /etc/network/if-%s.d",
		verbose ? "--verbose" : "", opt);

	(*exec)(buf); 

	return 1;
}
@ 

\subsubsection{[[iface_up()]], [[iface_down()]], [[iface_list()]], and [[iface_query()]]}

Our functions, then are:

<<exported symbols>>=
int iface_preup(interface_defn *iface);
int iface_postup(interface_defn *iface);
int iface_up(interface_defn *iface);
int iface_predown(interface_defn *iface);
int iface_postdown(interface_defn *iface);
int iface_down(interface_defn *iface);
int iface_list(interface_defn *iface);
int iface_query(interface_defn *iface);
@ 

<<execute functions>>=
int iface_preup(interface_defn *iface) {
	if (!iface->method->up(iface,check)) return -1;

	set_environ(iface, "start", "pre-up");
	if (!execute_options(iface,doit,"pre-up")) return 0;
	if (!execute_scripts(iface,doit,"pre-up")) return 0;

	return 1;
}

int iface_postup(interface_defn *iface) {
	if (!iface->method->up(iface,doit)) return 0;

	set_environ(iface, "start", "post-up");
	if (!execute_options(iface,doit,"up")) return 0;
	if (!execute_scripts(iface,doit,"up")) return 0;

	return 1;
}

int iface_up(interface_defn *iface) {
	int result = iface_preup(iface);
	if (result != 1) return result;
	return iface_postup(iface);
}
@ 

When bringing interface down, we check if there's [[ifup]] is still running
and send [[SIGTERM]] to terminate it.

<<terminate ifup if it is still running>>=
char pidfilename[100];
snprintf(pidfilename, sizeof(pidfilename), RUN_DIR "ifup-%s.pid",
	iface->real_iface);
FILE * pidfile = fopen(pidfilename, "r");
if (pidfile) {
	int pid;
	if (fscanf(pidfile, "%d", &pid)) {
		if (verbose) {
			fprintf(stderr, "Terminating ifup (pid %d)\n", pid);
		}
		kill((pid_t) -pid, SIGTERM);
	}
	fclose(pidfile);
	unlink(pidfilename);
}
@ 

<<execute functions>>=
int iface_predown(interface_defn *iface) {
	if (!no_act) {
	    <<terminate ifup if it is still running>>
	}

	if (!iface->method->down(iface,check)) return -1;

	set_environ(iface, "stop", "pre-down");
	if (!execute_scripts(iface,doit,"down")) return 0;
	if (!execute_options(iface,doit,"down")) return 0;

	return 1;
}

int iface_postdown(interface_defn *iface) {
	if (!iface->method->down(iface,doit)) return 0;

	set_environ(iface, "stop", "post-down");
	if (!execute_scripts(iface,doit,"post-down")) return 0;
	if (!execute_options(iface,doit,"post-down")) return 0;

	return 1;
}

int iface_down(interface_defn *iface) {
	int result = iface_predown(iface);
	if (result != 1) return result;
	return iface_postdown(iface);
}
@ 

<<execute functions>>=
int iface_list(interface_defn *iface) {
	printf("%s\n",iface->real_iface);
	return 0;
}
@

<<execute functions>>=
int iface_query(interface_defn *iface) {
	int i;
	for (i = 0; i < iface->n_options; i++) {
		printf("%s: %s\n",iface->option[i].name, iface->option[i].value);
	}
	return 0;
}
@

\subsection{Command Parsing}

All the above just leave one thing out: how the address family method's
configuration function gets back to calling [[doit()]]. This function
answers that question:

<<exported symbols>>=
int execute(char *command, interface_defn *ifd, execfn *exec);
@ 

At the somewhat abstract level, this is fairly trivial. The devil is
in the details of the parsing, which makes up the rest of the module.

<<execute functions>>=
int execute(char *command, interface_defn *ifd, execfn *exec) { 
	char *out;
	int ret;

	out = parse(command, ifd);
	if (!out) { return 0; }

	ret = (*exec)(out);

	free(out);
	return ret;
}
@ 

We'll need a basic parser function, which we'll call [[parse()]], to
make the appropriate substitutions into a command. It's probably worth
a note as to exactly what substitutions may be made:

\begin{itemize}

	\item Special characters can be escaped with a backslash. eg
	[[ls MoreThan80\%]].

	\item Variables can be substituted by including their name
	delimeted by percents. eg [[ls %directory%]].

	\item Optional components may be enclosed in double square
	brackets. Optional components will be included exactly when
	every variable referenced within exists. eg
	[[ls [[--color=%color%]]][[] %directory%]]. Optional components
	may be nested.

\end{itemize}

Most of the parsing is fairly straightforward -- basically, we keep an
output buffer, and add things to it as we stroll through the input
buffer: either the actual character we want, or whatever the value of
the variable we're looking at is, or whatever. The only particularly
complicated bit is how we deal with the optional sections, which will
be explained when we get to them.

<<execute function declarations>>=
static char *parse(char *command, interface_defn *ifd);
@ 

<<execute functions>>=
static char *parse(char *command, interface_defn *ifd) {
	<<variables local to parse>>

	while(*command) {
		switch(*command) {
			<<handle a token>>
		}
	}

	<<deal with error conditions>>

	<<return result>>
}
@

\subsubsection{Maintain output buffer}

So the first thing we need to do is actually write some code to deal
with the output buffer, which will need to be dynamically resized and
so on to take care of possibly long strings and what-not. It is the
caller's responsibility to [[free()]] this buffer. We'll maintain two
extra variables for convenience: who much memory we've allocated
[[len]], and where the next character should be stuck [[pos]].

<<variables local to parse>>=
char *result = NULL;
size_t pos = 0, len = 0;
@ 

This makes it pretty easy to return the result to the caller, too.

<<return result>>=
return result;
@

The main thing to be done to this buffer is to add characters or
strings to it. To deal with this, we'll make use of an [[addstr()]]
function that resizes the buffer as necessary, and appends a string to
it. So we can deal with single characters, and substrings in general,
we'll specify the string to be added as a pointer-length combination,
rather than as a [[NUL]] terminated string.

<<execute function declarations>>=
void addstr(char **buf, size_t *len, size_t *pos, char *str, size_t strlen);
@ 

<<execute functions>>=
void addstr(char **buf, size_t *len, size_t *pos, char *str, size_t strlen) {
	assert(*len >= *pos);
	assert(*len == 0 || (*buf)[*pos] == '\0');

	if (*pos + strlen >= *len) {
		char *newbuf;
		newbuf = realloc(*buf, *len * 2 + strlen + 1);
		if (!newbuf) {
			perror("realloc");
			exit(1); /* a little ugly */
		}
		*buf = newbuf;
		*len = *len * 2 + strlen + 1;
	}

	while (strlen-- >= 1) {
		(*buf)[(*pos)++] = *str;
		str++;
	}
	(*buf)[*pos] = '\0';
}
@ 

Given this, we can define our default behaviour for a character:

<<handle a token>>=
default:
	addstr(&result, &len, &pos, command, 1);
	command++;
	break;
@ 

\subsubsection{Escaped characters}

We can also deal pretty simply with escaped tokens. The only special
circumstance is if the [[\]] is at the very end of string. We don't
want buffer overflows afterall.

<<handle a token>>=
case '\\':
	if (command[1]) {
		addstr(&result, &len, &pos, command+1, 1);
		command += 2;
	} else {
		addstr(&result, &len, &pos, command, 1);
		command++;
	}
	break;
@ 

\subsubsection{Optional components}

Basically we keep track of each optional section we're in, whether
we've been unable to fill in any variables, and where we started
it. When we reach the end of an optional section, we check to see if
we were unable to fill in any variables, and, if so, we discard any
text we'd added within that block. This also allows us to neatly check
for any errors trying to fill in variables that aren't in optional
sections.

Basically what we'll do here is keep one stack to represent where the
various thingos started, and another to represent whether any
variables didn't exist. We'll use the bottom-most entry in the stack
to represent the entire command, and thus keep track of whether or not
we have to return an error because an undefined variable was used in a
non-optional part of the command.

<<constant definitions>>=
#define MAX_OPT_DEPTH 10
@ 

<<variables local to parse>>=
size_t old_pos[MAX_OPT_DEPTH] = {0};
int okay[MAX_OPT_DEPTH] = {1};
int opt_depth = 1;
@ 

Given this, when we encounter a double open bracket, we need to just
add the appropriate values to our stacks, and, similarly, when we
encounter a double close bracket, we simply need to pop the stack, and
see whether we need to move back or not, as well as taking care of an
possible errors, naturally. \emph{We probably could actually give
error messages here instead of just treating the brackets literally
when they might cause problems. But there doesn't seem much point,
really. --- aj}

<<handle a token>>=
case '[':
	if (command[1] == '[' && opt_depth < MAX_OPT_DEPTH) {
		old_pos[opt_depth] = pos;
		okay[opt_depth] = 1;
		opt_depth++;
		command += 2;
	} else {
		addstr(&result, &len, &pos, "[", 1);
		command++;
	}
	break;
@ 

<<handle a token>>=
case ']':
	if (command[1] == ']' && opt_depth > 1) {
		opt_depth--;
		if (!okay[opt_depth]) {
			pos = old_pos[opt_depth];
			result[pos] = '\0';
		}
		command += 2;
	} else {
		addstr(&result, &len, &pos, "]", 1);
		command++;
	}
	break;
@

Finally, at the end of the function, the stacks can be left in an
unacceptable state --- either one of the optional blocks was never
closed, or an undefined variable was used elsewhere. We'll note these
circumstances by returning [[NULL]] and setting [[errno]].

<<execute headers>>=
#include <errno.h>
@

<<constant definitions>>=
#define EUNBALBRACK 10001
#define EUNDEFVAR   10002
@ 

<<deal with error conditions>>=
if (opt_depth > 1) {
	errno = EUNBALBRACK;
	free(result);
	return NULL;
}

if (!okay[0]) {
	errno = EUNDEFVAR;
	free(result);
	return NULL;
}
@ 

\subsubsection{Variables}

Dealing with variables is comparatively fairly simple. We just need to
find the next percent, and see if whatever's in-between is a variable,
and, if so, get it's value.

<<constant definitions>>=
#define MAX_VARNAME    32
#define EUNBALPER   10000
@ 

<<handle a token>>=
case '%':
{
	<<variables local to handle percent token>>
	char *varvalue;

	<<determine variable name>>

	<<get [[varvalue]]>>

	if (varvalue) {
		<<replace the character if needed>>
		addstr(&result, &len, &pos, varvalue, strlen(varvalue));
		free(varvalue);
	} else {
		if (opt_depth == 1) {
			fprintf(stderr, "Missing required variable: %.*s\n", namelen, command);
		}
		okay[opt_depth - 1] = 0;
	}

	<<move to token after closing percent>>

	break;
}
@ 

We don't do anything particularly clever dealing with the next percent
--- just a pointer to the appropriate character.

<<variables local to handle percent token>>=
char *nextpercent;
size_t namelen;
char pat = 0, rep = 0;
@ 

We support doing a simple replacement in the values returned by [[get_var()]].
When the syntax `\verb!%var/p/r/%!' is used, the first occurence of the
symbol `\verb!p!' will be replaced by `\verb!r!' when the substitution
is done. Currently, both pattern and replacement string can be one character
length only.

<<determine variable name>>=
command++;
nextpercent = strchr(command, '%');
namelen = nextpercent - command;
if (!nextpercent) {
	errno = EUNBALPER;
	free(result);
	return NULL;
}
/* %var/p/r% */
if (*(nextpercent - 4) == '/') {
	pat = *(nextpercent - 3);
	rep = *(nextpercent - 1);
	namelen -= 4;
}
@ 

<<replace the character if needed>>=
char * position = strchr(varvalue, pat);
if (position) {
	*position = rep;
}
@ 

<<move to token after closing percent>>=
command = nextpercent + 1;
@

The slightly tricky thing we do here is use a [[strncmpz]] function,
which allows us to check that a string represented by a [[char*]] and
a length is the same as a [[NUL]] terminated string.

<<exported symbols>>=
int strncmpz(char *l, char *r, size_t llen);
@ 

<<execute functions>>=
int strncmpz(char *l, char *r, size_t llen) {
	int i = strncmp(l, r, llen);
	if (i == 0)
		return -r[llen];
	else
		return i;
}
@ 

Given the above, the implementation of the [[get_var()]] function to
lookup the value of a variable, is reasonably straight forward. The
only issue to address in this function is a pseudo-variable
[[%iface%]].

<<exported symbols>>=
char *get_var(char *id, size_t idlen, interface_defn *ifd);
@ 

<<execute functions>>=
char *get_var(char *id, size_t idlen, interface_defn *ifd) {
	int i;

	if (strncmpz(id, "iface", idlen) == 0) {
		return strdup(ifd->real_iface);
	}

	{
		for (i = 0; i < ifd->n_options; i++) {
			if (strncmpz(id, ifd->option[i].name, idlen) == 0) {
				if (!ifd->option[i].value) {
				    return NULL;
				}
				if (strlen(ifd->option[i].value) > 0) {
					return strdup(ifd->option[i].value);
				} else {
					return NULL;
				}
			}
		}
	}

	return NULL;
}
@ 

Which means we can finish of the chunk, thus:

<<get [[varvalue]]>>=
varvalue = get_var(command, namelen, ifd);
@ 

We also define an exported [[var_true()]], [[var_set()]] and [[var_set_anywhere()]] functions
to allow methods to have lines of code that are conditional on the value
of a variable.

<<exported symbols>>=
int var_true(char *id, interface_defn *ifd);
int var_set(char *id, interface_defn *ifd);
int var_set_anywhere(char *id, interface_defn *ifd);
@

<<execute functions>>=
int var_true(char *id, interface_defn *ifd) {
	char *varvalue;

	varvalue = get_var(id, strlen(id), ifd);
	if (varvalue) {
		if (atoi(varvalue) ||
			strcasecmp(varvalue, "on") == 0 ||
			strcasecmp(varvalue, "true") == 0 ||
			strcasecmp(varvalue, "yes") == 0)
		{
			free(varvalue);
			return 1;
		} else {
			free(varvalue);
			return 0;
		}
	} else
		return 0;
}

int var_set(char *id, interface_defn *ifd) {
	char *varvalue;

	varvalue = get_var(id, strlen(id), ifd);
	if (varvalue) {
		free(varvalue);
		return 1;
	} else {
		return 0;
	}
}

int var_set_anywhere(char *id, interface_defn *ifd) {
	char *varvalue;
	interface_defn *currif;

	for (currif = defn->ifaces; currif; currif = currif->next) {
		if (strcmp(ifd->logical_iface, currif->logical_iface) == 0) {
			varvalue = get_var(id, strlen(id), currif);
			if (varvalue) {
				free(varvalue);
				return 1;
			}
		}
	}
	return 0;
}
@

\subsection{Mapping Scripts}

Doing a mapping is moderately complicated, since we need to pass a
fair bit of stuff to the script. The way we'll do this is via a
mixture of command line arguments, and [[stdin]]: basically, we'll
pass all the mapping variables from the interfaces file via [[stdin]],
and anything else necessary will be a command line argument. The
script will be expected to exit successfully with the appropriate
logical interface as the first line of [[stdout]] if it made a match,
or exit unsuccessfully (error code [[1]], eg) otherwise.

<<exported symbols>>=
int run_mapping(char *physical, char *logical, int len, mapping_defn *map);
@ 

<<execute functions>>=
int run_mapping(char *physical, char *logical, int len, mapping_defn *map) {
	FILE *in, *out;
	int i, status;
	pid_t pid;

	<<execute the mapping script>>
	<<send input to mapping script>>
	<<wait for mapping script to finish>>
	<<check output from mapping script>>

	return 1;
}
@ 

The latter options here are fairly straightforward, given some Unix
knowledge.

<<send input to mapping script>>=
for (i = 0; i < map->n_mappings; i++) {
	fprintf(in, "%s\n", map->mapping[i]);
}
fclose(in);
@ 

<<wait for mapping script to finish>>=
waitpid(pid, &status, 0);
@ 

<<check output from mapping script>>=
if (WIFEXITED(status) && WEXITSTATUS(status) == 0) {
	if (fgets(logical, len, out)) {
		char *pch = logical + strlen(logical) - 1;
		while (pch >= logical && isspace(*pch)) 
			*(pch--) = '\0';
	}
}
fclose(out);	
@ 

Slightly more complicated is setting up the child process and grabbing
its [[stdin]] and [[stdout]]. Unfortunately we can't just use
[[popen()]] for this, since it'll only allow us to go one way. So,
instead we'll write our own [[popen()]]. It'll look like:

<<execute headers>>=
#include <stdarg.h>
@ 

<<execute function declarations>>=
static int popen2(FILE **in, FILE **out, char *command, ...);
@ 

The varargs component will be the arguments, as per [[execl()]], and
the return value will be the [[PID]] if the call was successful, or 0
otherwise.

As such, we will be able to execute the script thusly:

<<execute the mapping script>>=
pid = popen2(&in, &out, map->script, physical, NULL);
if (pid == 0) {
	return 0;
}
@ 

Writing [[popen2()]] is an exercise in Unix arcana.

<<execute headers>>=
#include <unistd.h>
#include <sys/wait.h>
@ 

<<execute functions>>=
static int popen2(FILE **in, FILE **out, char *command, ...) {
	va_list ap;
	char *argv[11] = {command};
	int argc;
	int infd[2], outfd[2];
	pid_t pid;

	argc = 1;
	va_start(ap, command);
	while((argc < 10) && (argv[argc] = va_arg(ap, char*))) {
		argc++;
	}
	argv[argc] = NULL; /* make sure */
	va_end(ap);

	if (pipe(infd) != 0) return 0;
	if (pipe(outfd) != 0) {
		close(infd[0]); close(infd[1]);
		return 0;
	}

	fflush(NULL);
	switch(pid = fork()) {
		case -1: /* failure */
			close(infd[0]); close(infd[1]);
			close(outfd[0]); close(outfd[1]);
			return 0;
		case 0: /* child */
			/* release the current directory */
			chdir("/");
			dup2(infd[0], 0);
			dup2(outfd[1], 1);
			close(infd[0]); close(infd[1]);
			close(outfd[0]); close(outfd[1]);
			execvp(command, argv);
			exit(127);
		default: /* parent */
			*in = fdopen(infd[1], "w");
			*out = fdopen(outfd[0], "r");
			close(infd[0]);	close(outfd[1]);
			return pid;
	}
	/* unreached */
}
@

\section{The Driver}

The final C module we'll have is the one with the [[main()]]
function. It's put together in a fairly straightforward way.

<<main.c>>=
<<main headers>>
<<main global variables>>
<<main function declarations>>
<<main functions>>
<<main>>
@ 

Equally, there's nothing particularly special about our headers.

<<main headers>>=
#include <stdio.h>
#include <stdlib.h>
#include <string.h>
#include <ctype.h>
#include <errno.h>
#include <assert.h>

#include "header.h"
@

Now, after all the above modules, our main program doesn't have too
much to do: it just has to interpret arguments and coordinate the
intervening modules. Since we're being ``smart'', as well as parsing
arguments, we'll decide whether the interface is going up or down
depending on whether we were called as [[ifup]] or [[ifdown]], or if we're
querying /etc/network/interfaces (when called as [[ifquery]]).

<<main>>=
int main(int argc, char **argv) {
	<<variables local to main>>

	<<ensure environment is sane>>

	<<parse command name or die>>
	<<parse arguments>>

	<<read interfaces files or die>>

	<<run commands for appropriate interfaces>>

	return 0;
}
@

\subsection{Check the Environment}

In the earlier code we assume that we have stdin, stdout and stderr all
available. We need to assure that, just in case:

<<main headers>>=
#include <unistd.h>
#include <fcntl.h>
@

<<ensure environment is sane>>=
{
	int i;
	for (i = 0; i <= 2; i++) {
		if (fcntl(i, F_GETFD) == -1) {
			if (errno == EBADF && open("/dev/null", 0) == -1) {
				fprintf(stderr,
					"%s: fd %d not available; aborting\n",
					argv[0], i);
				exit(2);
			} else if (errno == EBADF) {
				errno = 0; /* no more problems */
			} else {
				/* some other problem -- eeek */
				perror(argv[0]);
				exit(2);
			}
		}
	}
}
@

\subsection{Configuring or Deconfiguring?}

So the very first real thing we need to do is parse the command name. To
do this, we'll obviously need to work out somewhere to store the result. A
reasonable thing to do here is just to keep a function pointer about,
which will point to one of the previously defined [[iface_up]] or
[[iface_down]] functions, depending on which should be used on the
specified interfaces.

<<variables local to main>>=
int (*cmds)(interface_defn *) = NULL;
@

So given this, we can just:

<<parse command name or die>>=
{
	char *command;

	<<set [[command]] to the base of the command name>>
	<<set [[cmds]] based on [[command]] or die>>
}
@

And fill out each component in the reasonably obvious manner of:

<<set [[command]] to the base of the command name>>=
if ((command = strrchr(argv[0],'/'))) {
	command++; /* first char after / */
} else {
	command = argv[0]; /* no /'s in argv[0] */
}
@

<<set [[cmds]] based on [[command]] or die>>=
if (strcmp(command, "ifup")==0) {
	cmds = iface_up;
} else if (strcmp(command, "ifdown")==0) {
	cmds = iface_down;
} else if (strcmp(command, "ifquery")==0) {
	cmds = iface_query;
	no_act = 1;
} else {
	fprintf(stderr,"This command should be called as ifup, ifdown, or ifquery\n");
	exit(1);
}
@ 

In addition, since our later behaviour varies depending on whether we're
bringing interfaces up, taking them down, or querying /etc/network/interfaces,
we'll define four chunks to assist with this, namely:

<<we're querying an interface config>>=
(cmds == iface_query)
@

<<we're listing known interfaces>>=
(cmds == iface_list)
@

<<we're bringing interfaces up>>=
(cmds == iface_up)
@

<<we're taking interfaces down>>=
(cmds == iface_down)
@

The [[--allow]] option lets us limit the interfaces ifupdown will act on.
It's implemented by having an [[allow_class]] that tells us which class
of interfaces we're working with, and skipping interfaces that aren't
in that class, like so:

<<we're limiting to [[--allow]]ed interfaces>>=
(allow_class != NULL)

<<find [[iface]] in [[allow_class]] or [[continue]]>>=
{
	int i;
	allowup_defn *allowup = find_allowup(defn, allow_class);
	if (allowup == NULL)
		continue;

	for (i = 0; i < allowup->n_interfaces; i++) {
		if (strcmp(allowup->interfaces[i], iface) == 0)
			break;
	}
	if (i >= allowup->n_interfaces)
		continue;
}
@

Finally, the behaviour might vary depending on whether we are 
excluding this interface or not. Notice that
the exclude option can use a full interface name or substrings that
match interfaces. A user could easily have unexpected behaviour
if he uses a small string to do the match:

<<we're [[--exclude]]ing this interface>>=
(excludeints != 0 && match_patterns(iface, excludeints, excludeint))
@

[[match_patterns]] function goes through exclusion patterns and returns [[true]] if it finds at least one match.

<<main function declarations>>=
bool match_patterns(char * string, int argc, char * argv[]);
@

It's implemented using [[fnmatch]] function, so we can use shell globs.

<<main functions>>=
bool match_patterns(char * string, int argc, char * argv[]) {
	if (!argc || !argv || !string) return false;
	int i;
	for (i = 0; i < argc; i++) {
		if (fnmatch(argv[i], string, 0) == 0) {
			return true;
		}
	}
	return false;
}
@

\subsection{Argument Handling}

Okay, so next on our agenda is argument handling.

We'll do argument handling via the GNU [[getopt]] function, which
means we have to include the appropriate header, and define a cute
little structure to represent our long options:

<<main headers>>=
#include <getopt.h>
@

<<variables local to main>>=
struct option long_opts[] = {
	{"help",        no_argument,       NULL, 'h'},
	{"version",     no_argument,       NULL, 'V'},
	{"verbose",     no_argument,       NULL, 'v'},
	{"all",         no_argument,       NULL, 'a'},
	{"allow",	required_argument, NULL,  3 },
	{"interfaces",  required_argument, NULL, 'i'},
	{"exclude",     required_argument, NULL, 'X'},
	{"no-act",      no_argument,       NULL, 'n'},
	{"no-mappings", no_argument,       NULL,  1 },
	{"no-scripts",  no_argument,       NULL,  4 },
	{"no-loopback", no_argument,       NULL,  5 },
	{"force",       no_argument,       NULL,  2 },
	{"option",      required_argument, NULL, 'o'},
	{"list",        no_argument,       NULL, 'l'},
	{0,0,0,0}
};
@ 

The usual way of dealing with options then is to have a variable to store
the various things. The only special note here is that we need to export
[[no_act]] and [[verbose]] to the [[execute]] module.

<<exported symbols>>=
extern int no_act;
extern int verbose;
extern int run_scripts;
extern bool no_loopback;
@

<<constant definitions>>=
#ifndef RUN_DIR
#define RUN_DIR "/run/network/"
#endif

#ifndef LO_IFACE
#define LO_IFACE "lo"
#endif
@ 

<<main global variables>>=
int no_act = 0;
int run_scripts = 1;
int verbose = 0;
bool no_loopback = false;
char *statefile = RUN_DIR "ifstate";
char *tmpstatefile = RUN_DIR ".ifstate.tmp";
@

<<variables local to main>>=
int do_all = 0;
int run_mappings = 1;
int force = 0;
int list = 0;
char *allow_class = NULL;
char *interfaces = "/etc/network/interfaces";
char **excludeint = NULL;
int excludeints = 0;
variable *option = NULL;
int n_options = 0;
int max_options = 0;
@ 

We'll also have two helper functions to display usage information,
like so:

<<main function declarations>>=
static void usage(char *execname);
static void help(char *execname, int (*cmds)(interface_defn *));
static void version(char *execname);
@ 

<<main functions>>=
static void usage(char *execname) {
	fprintf(stderr, "%s: Use --help for help\n", execname);
	exit(1);
}
@ 

<<main functions>>=
static void version(char *execname) {
	printf("%s version " IFUPDOWN_VERSION "\n", execname);
	printf("Copyright (c) 1999-2007 Anthony Towns\n\n");
	printf(

"This program is free software; you can redistribute it and/or modify\n"
"it under the terms of the GNU General Public License as published by\n"
"the Free Software Foundation; either version 2 of the License, or (at\n"
"your option) any later version.\n"

	);
	exit(0);
}
@ 

<<main functions>>=
static void help(char *execname, int (*cmds)(interface_defn *)) {
	printf("Usage: %s <options> <ifaces...>\n", execname);
	if (<<we're listing known interfaces>>
	    || <<we're querying an interface config>>)
		printf("       %s <options> --list\n", execname);
	printf("\n");
	printf("Options:\n");
	printf("\t-h, --help\t\tthis help\n");
	printf("\t-V, --version\t\tcopyright and version information\n");
	printf("\t-a, --all\t\tprocess all interfaces marked \"auto\"\n");
	printf("\t--allow CLASS\t\tignore non-\"allow-CLASS\" interfaces\n");
	printf("\t-i, --interfaces FILE\tuse FILE for interface definitions\n");
	printf("\t-X, --exclude PATTERN\texclude interfaces from the list of\n\t\t\t\tinterfaces to operate on by a PATTERN\n");
	if (!<<we're listing known interfaces>>
	    && !<<we're querying an interface config>>)
		printf("\t-n, --no-act\t\tprint out what would happen, but don't do it\n");
	printf("\t\t\t\t(note that this option doesn't disable mappings)\n");
	printf("\t-v, --verbose\t\tprint out what would happen before doing it\n");
	printf("\t-o OPTION=VALUE\t\tset OPTION to VALUE as though it were in\n");
	printf("\t\t\t\t/etc/network/interfaces\n");
	printf("\t--no-mappings\t\tdon't run any mappings\n");
	printf("\t--no-scripts\t\tdon't run any hook scripts\n");
	printf("\t--no-loopback\t\tdon't act specially on the loopback device\n");
	if (!<<we're listing known interfaces>>
	    && !<<we're querying an interface config>>)
		printf("\t--force\t\t\tforce de/configuration\n");
	if (<<we're listing known interfaces>>
	    || <<we're querying an interface config>>)
		printf("\t--list\t\t\tlist all matching known interfaces\n");
	exit(0);
}
@ 

Now, the meat of argument parsing is done with [[getopt()]] and a
[[switch]], like so:

<<parse arguments>>=
for(;;) {
	int c;
	c = getopt_long(argc, argv, "X:s:i:o:hVvnal", long_opts, NULL);
	if (c == EOF) break;

	switch(c) {
		<<[[getopt]] possibilities>>
	}
}

<<check unreasonable arguments>>
@ 

Now, our [[getopt]] possibilities are basically each option, or something
really bad. Actual interface names are automagically collected at the end
by [[getopt()]].So first, the legitimate cases get handled:

<<[[getopt]] possibilities>>=
case 'i':
	interfaces = strdup(optarg);
	break;
@ 
<<[[getopt]] possibilities>>=
case 'v':
	verbose = 1;
	break;
@ 
<<[[getopt]] possibilities>>=
case 'a':
	do_all = 1;
	break;
@ 
<<[[getopt]] possibilities>>=
case 3:
	allow_class = strdup(optarg);
	break;
@ 
<<[[getopt]] possibilities>>=
case 'n':
        if (<<we're listing known interfaces>> || <<we're querying an interface config>>)
		usage(argv[0]);
	no_act = 1;
	break;
@ 
<<[[getopt]] possibilities>>=
case 1:
	run_mappings = 0;
	break;
@ 
<<[[getopt]] possibilities>>=
case 4:
	run_scripts = 0;
	break;
@ 
<<[[getopt]] possibilities>>=
case 5:
	no_loopback = true;
	break;
@ 
<<[[getopt]] possibilities>>=
case 2:
        if (<<we're listing known interfaces>> || <<we're querying an interface config>>)
		usage(argv[0]);
	force = 1;
	break;
@
<<[[getopt]] possibilities>>=
case 'X':
	/* */
	excludeints++;
	excludeint = realloc(excludeint, excludeints * sizeof(char *));
	if (excludeint == NULL) {
		char * filename = argv[0];
		<<report internal error and die>>
	}
	excludeint[excludeints - 1] = strdup(optarg);
	break;
@ 
<<[[getopt]] possibilities>>=
case 'o':
{
	char *name = strdup(optarg);
	char *val = strchr(name, '=');
	if (val == NULL) {
		fprintf(stderr, "Error in --option \"%s\" -- no \"=\" character\n",
			optarg);
		exit(1);
	}
	*val++ = '\0';

	if (strcmp(name, "post-up") == 0) {
		strcpy(name, "up");
	}
	if (strcmp(name, "pre-down") == 0) {
		strcpy(name, "down");
	}
	
	set_variable(argv[0], name, val, &option, &n_options, &max_options);
	free(name);

	break;
}
@ 
<<[[getopt]] possibilities>>=
case 'l':
        if (!<<we're querying an interface config>>)
		usage(argv[0]);
	list = 1;
	cmds = iface_list;
	break;
@ 

And we also have a help option and a version option:

<<[[getopt]] possibilities>>=
case 'h':
	help(argv[0],cmds);
	break;
@ 
<<[[getopt]] possibilities>>=
case 'V':
	version(argv[0]);
	break;
@ 

And we also have the possibility that the user is just making up
options:

<<[[getopt]] possibilities>>=
default:
	usage(argv[0]);
	break;
@

After all that there are still some things that can be a bit weird. We
can be told either to act on all interfaces (except the noauto ones),
or we can be told to act on specific interface. We won't accept been
told to do both, and we won't accept not being told to do one or the
other. We can test these two cases as follows:

<<check unreasonable arguments>>=
if (argc - optind > 0 && (do_all || list)) {
	usage(argv[0]);
}
@ 

<<check unreasonable arguments>>=
if (argc - optind == 0 && !do_all && !list) {
	usage(argv[0]);
}
@ 

<<check unreasonable arguments>>=
if (do_all && <<we're querying an interface config>>) {
	usage(argv[0]);
}
@ 

\subsection{Reading the Interfaces File}

Since this has all been covered in a previous section, this is pretty
trivial.

<<exported symbols>>=
extern interfaces_file *defn;
@ 

<<main global variables>>=
interfaces_file *defn;
@ 

<<read interfaces files or die>>=
defn = read_interfaces(interfaces);
if ( !defn ) {
	fprintf(stderr, "%s: couldn't read interfaces file \"%s\"\n",
		argv[0], interfaces);
	exit(1);
}
@

\subsection{Execution}

A broad overview of what we'll actually be doing is as follows:

<<run commands for appropriate interfaces>>=
<<determine target interfaces>>
<<run pre-up and pre-down scripts for all>>
{
	int i;
	for (<<each target interface, [[i]]>>) {
		char iface[80], liface[80];
		const char *current_state;

		<<initialize [[iface]] to [[i]]th target interface>>
		current_state = read_state(argv[0], iface);
		if (!force) {
			<<check ifupdown state (possibly [[continue]])>>
		}

		if (<<we're limiting to [[--allow]]ed interfaces>>) {
			<<find [[iface]] in [[allow_class]] or [[continue]]>>
		}

		if (<<we're [[--exclude]]ing this interface>>)  
			continue;

		bool have_mapping = false;
		if ((<<we're bringing interfaces up>> && run_mappings) || <<we're querying an interface config>>) {
			<<run mappings>>
		}

		<<bring interface up/down and update ifupdown state>>
	}
}
<<run post-up and post-down scripts for all>>
@

We'll leave determining the appropriate target interfaces and dealing
with the state until a little later. That leaves us with covering running
the mappings and bringing the interface up or taking it down.

Mappings are dealt with like so:

<<run mappings>>=
{
	mapping_defn *currmap;
	for (currmap = defn->mappings; currmap; currmap = currmap->next) {
		int i;
		for (i = 0; i < currmap->n_matches; i++) {
			<<[[continue]] unless mapping matches>>
			<<run mapping>>
			break;
		}
	}
}
@

We check if mappings match by using shell globs, so we'll need a new header
to take care of that.

<<main headers>>=
#include <fnmatch.h>
@ 

<<[[continue]] unless mapping matches>>=
if (fnmatch(currmap->match[i], liface, 0) != 0)
	continue;
@

Actually running a mapping is fairly straightforward, thanks to our
previous handywork.

<<run mapping>>=
if (<<we're querying an interface config>> && !run_mappings) {
	if (verbose) {
		fprintf(stderr, "Not running mapping scripts for %s\n",
			liface);
	}
	have_mapping = true;
	break;
}
if (verbose) {
	fprintf(stderr, "Running mapping script %s on %s\n",
		currmap->script, liface);
}
run_mapping(iface, liface, sizeof(liface), currmap);
@

Bringing an interface up or taking it down can be done thusly:

<<bring interface up/down and update ifupdown state>>=
{
	interface_defn *currif;
	int okay = 0;
	int failed = 0; 

	<<update ifupdown state>>

	if (<<we're listing known interfaces>>) {
	    for (currif = defn->ifaces; currif; currif = currif->next) {
		    if (strcmp(liface, currif->logical_iface) == 0) {
			okay = 1;
		    }
	    }
	    if (!okay) {
		    mapping_defn *currmap;
		    for (currmap = defn->mappings; currmap; currmap = currmap->next) {
			    int i;
			    for (i = 0; i < currmap->n_matches; i++) {
				    <<[[continue]] unless mapping matches>>
				    okay = 1;
				    break;
			    }
		    }
	    }
	    if (okay) {
		    currif = defn->ifaces;
		    currif->real_iface = iface;
		    cmds(currif);
		    currif->real_iface = NULL;
	    }
	    okay = 0;
	    continue;
	}

	for (currif = defn->ifaces; currif; currif = currif->next) {
		if (strcmp(liface, currif->logical_iface) == 0) {
			<<configure the link>>
			okay = 1;

			<<add default options to [[currif]]>>

			<<add options from command line to [[currif]]>>

			currif->real_iface = iface;

			<<convert options>>

			<<run commands for [[currif]]; set [[failed]] on error>>

			currif->real_iface = NULL;

			if (failed) break;
			/* Otherwise keep going: this interface may have
			 * match with other address families */
		}
	}

	<<deconfigure the link>>

	if (!okay && <<we're querying an interface config>>) {
		if (!run_mappings) {
			if (have_mapping) {
				okay = 1;
			}
		}
		if (!okay) {
			<<report unknown interface and die>>
		}
	}

	if (!okay && !force) {
		fprintf(stderr, "Ignoring unknown interface %s=%s.\n", 
			iface, liface);
		update_state (argv[0], iface, NULL);
	} else {
		<<update ifupdown state>>
	}
}
@

<<configure the link>>=
	if (!okay && <<we're bringing interfaces up>>) {
		interface_defn link = {
		    .real_iface = iface,
		    .logical_iface = liface,
		    .max_options = 0,
		    .address_family = &addr_link,
		    .method = &(addr_link.method[0]),
		    .n_options = 0,
		    .option = NULL
		};
		convert_variables(argv[0], link.method->conversions, &link);

		if (!link.method->up(&link, doit)) break;
		if (link.option) free(link.option);
	}
@ 

<<deconfigure the link>>=
	if (okay && <<we're taking interfaces down>>) {
		interface_defn link = {
		    .real_iface = iface,
		    .logical_iface = liface,
		    .max_options = 0,
		    .address_family = &addr_link,
		    .method = &(addr_link.method[0]),
		    .n_options = 0,
		    .option = NULL
		};
		convert_variables(argv[0], link.method->conversions, &link);

		if (!link.method->down(&link, doit)) break;
		if (link.option) free(link.option);
	}
@ 

<<create pidfile>>=
{
	char * command;
	<<set [[command]] to the base of the command name>>
	snprintf(pidfilename, sizeof(pidfilename), RUN_DIR "%s-%s.pid",
		command, currif->real_iface);
	if (!no_act) {
		FILE * pidfile = fopen(pidfilename, "w");
		if (pidfile) {
			fprintf(pidfile, "%d", getpid());
			fclose(pidfile);
		} else {
			fprintf(stderr, 
				"%s: failed to open pid file %s: %s\n",
				command, pidfilename, strerror(errno));
		}
	}
}
@ 

<<remove pidfile>>=
	if (!no_act) {
		unlink(pidfilename);
	}
@ 

<<run commands for [[currif]]; set [[failed]] on error>>=
{
	if (verbose) {
		fprintf(stderr, "%s interface %s=%s (%s)\n", 
			<<we're querying an interface config>> ? "Querying" :
			"Configuring",
			iface, liface, currif->address_family->name);
	}

	char pidfilename[100];
	<<create pidfile>>

	switch(cmds(currif)) {
	    case -1:
		fprintf(stderr, "Missing required configuration variables for interface %s/%s.\n", 
			liface, currif->address_family->name);
		failed = 1;
		break;
	    case 0:
		failed = 1;
		break;
		/* not entirely successful */
	    case 1:
	    	failed = 0;
		break;
		/* successful */
	    default:
	    	fprintf(stderr, "Unexpected value when configuring interface %s/%s; considering it failed.\n", 
			liface, currif->address_family->name);
	    	failed = 1;
		/* what happened here? */
	}

	<<remove pidfile>>
}
@ 

Before bringing interfaces up or putting them down, we may want to call some
scripts interested in being notified when all the interfaces are going to
be brought up:

<<run pre-up and pre-down scripts for all>>=
interface_defn meta_iface = {
    .next = NULL,
    .real_iface = "--all",
    .address_family = &addr_meta,
    .method = &(addr_meta.method[0]),
    .automatic = 1,
    .max_options = 0,
    .n_options = 0,
    .option = NULL
};

if (do_all) {
    meta_iface.logical_iface = allow_class ? allow_class : "auto";

    int okay = 1;
    if (<<we're bringing interfaces up>>) {
	okay = iface_preup(&meta_iface);
    }
    if (<<we're taking interfaces down>>) {
	okay = iface_predown(&meta_iface);
    }
    if (!okay) {
	fprintf(stderr, "%s: pre-%s script failed.\n", argv[0], &argv[0][2]);
	exit(1);
    }
}
@ 

<<run post-up and post-down scripts for all>>=
if (do_all) {
    int okay = 1;
    if (<<we're bringing interfaces up>>) {
	okay = iface_postup(&meta_iface);
    }
    if (<<we're taking interfaces down>>) {
	okay = iface_postdown(&meta_iface);
    }
    if (!okay) {
	fprintf(stderr, "%s: post-%s script failed.\n", argv[0], &argv[0][2]);
	exit(1);
    }
}
@ 

Before adding any options from the command line we set the default options if
they've not been set already:

<<add default options to [[currif]]>>=
{
	option_default *o;
	for (o = currif->method->defaults; o && o->option && o->value; o++) {
		int j;
		int found = 0;
		for (j = 0; j < currif->n_options; j++) {
			if (strcmp(currif->option[j].name, 
			           o->option) == 0) 
			{
				found = 1;
				break;
			}
		}
		if (!found) {
			set_variable(argv[0], o->option, o->value,
				&currif->option, &currif->n_options, 
				&currif->max_options);
		}
	}
}
@

Adding the options from the command line is tedious, but simple:

<<add options from command line to [[currif]]>>=
{
	int i;
	for (i = 0; i < n_options; i++) {
		if (option[i].value[0] == '\0') {
			<<remove [[option[i]]] from [[currif]]>>
		} else {
			<<add [[option[i]]] to [[currif]]>>
		}
	}
}
@

Convert the options:

<<convert options>>=
{
	convert_variables(argv[0], currif->method->conversions, currif);
}
@

<<add [[option[i]]] to [[currif]]>>=
{
	set_variable(argv[0], option[i].name, option[i].value,
		&currif->option, &currif->n_options, 
		&currif->max_options);
}
@

<<remove [[option[i]]] from [[currif]]>>=
{
	if (strcmp(option[i].name, "pre-up") != 0
    	    && strcmp(option[i].name, "up") != 0
	    && strcmp(option[i].name, "down") != 0
	    && strcmp(option[i].name, "post-down") != 0)
	{
		int j;
		for (j = 0; j < currif->n_options; j++) {
			if (strcmp(currif->option[j].name, 
			           option[i].name) == 0) 
			{
				currif->n_options--;
				break;
			}
		}
		for (; j < currif->n_options; j++) {
			option[j].name = option[j+1].name;
			option[j].value = option[j+1].value;
		}
	} else {
		/* do nothing */
	}
}
@

\subsection{Target Interfaces}

So, if we're going to actually do something, we should probably figure
out exactly what we're going to do it to. So, we need to know the set
of interfaces we're going to hax0r. This is just an array of interfaces,
either [[physical_iface]] or [[physical_iface=logical_iface]].

<<variables local to main>>=
int n_target_ifaces;
char **target_iface;
@ 

<<each target interface, [[i]]>>=
i = 0; i < n_target_ifaces; i++
@

We initialise this based on our command line arguments.

<<determine target interfaces>>=
if (do_all || list) {
	if (<<we're listing known interfaces>>
            || <<we're bringing interfaces up>>) {
		allowup_defn *autos = find_allowup(defn, allow_class ? allow_class : "auto");
		target_iface = autos ? autos->interfaces : NULL;
		n_target_ifaces = autos ? autos->n_interfaces : 0;
	} else if (<<we're taking interfaces down>>) {
		read_all_state(argv[0], &target_iface, &n_target_ifaces);
	} else {
		fprintf(stderr, "%s: can't tell if interfaces are going up or down\n", argv[0]);
		exit(1);
	}	
} else {
	target_iface = argv + optind;
	n_target_ifaces = argc - optind;
}
@ 

<<initialize [[iface]] to [[i]]th target interface>>=
strncpy(iface, target_iface[i], sizeof(iface));
iface[sizeof(iface)-1] = '\0';

{
	char *pch;
	if ((pch = strchr(iface, '='))) {
		*pch = '\0';
		strncpy(liface, pch+1, sizeof(liface));
		liface[sizeof(liface)-1] = '\0';
	} else {
		strncpy(liface, iface, sizeof(liface));
		liface[sizeof(liface)-1] = '\0';
	}
}
@ 

\subsection{State}

Since it's generally not feasible to rerun a mapping script after an
interface is configured (since a mapping script may well bring the
interface down while it's investigating matters), we need to maintain a
statefile between invocations to keep track of which physical interfaces
were mapped to which logical ones.  This file also serves to record
which interfaces have been configured so far, and which haven't.  It
is stored in [[/run/network/ifstate]].

Because different interfaces may be brought up and down at the same time,
it's important that all updates to the state file are atomic and that we
aren't confused by any changes made by another running process.  For this
reason we use functions to examine or modify the state file at the point
necessary rather than holding it all in memory.

<<main function declarations>>=
static const char *read_state(const char *argv0, const char *iface);
static void read_all_state(const char *argv0, char ***ifaces, int *n_ifaces);
static void update_state(const char *argv0, const char *iface, const char *liface);
@ 

The first of these functions reads the state file to look for an interface
and returns the current state of it as a pointer to a static buffer which
should be copied if it's needed for any duration.  If the interface has no
current state (ie. is down) then NULL is returned.

<<main functions>>=
static const char *
read_state (const char *argv0, const char *iface)
{
	char *ret = NULL;

	<<open ifupdown state>>

	while((p = fgets(buf, sizeof buf, state_fp)) != NULL) {
		<<parse ifupdown state line>>

		if (strncmp(iface, pch, strlen(iface)) == 0) {
			if (pch[strlen(iface)] == '=') {
				ret = pch + strlen(iface) + 1;
				break;
			}
		}
	}

	<<close ifupdown state>>

	return ret;
}
@

The second of these functions is a variant on the above used to grab a list
of all currently up interfaces so we can tear them all down. Also, we reverse
the order of the interfaces so we don't break complex
configurations with VLAN devices.

<<main functions>>=
static void
read_all_state (const char *argv0, char ***ifaces, int *n_ifaces)
{
	int i;
	<<open ifupdown state>>

	*n_ifaces = 0;
	*ifaces = NULL;

	while((p = fgets(buf, sizeof buf, state_fp)) != NULL) {
		<<parse ifupdown state line>>

		(*n_ifaces)++;
		*ifaces = realloc (*ifaces, sizeof (**ifaces) * *n_ifaces);
		(*ifaces)[(*n_ifaces)-1] = strdup (pch);
	}

	for (i = 0; i < ((*n_ifaces)/2); i++) {
		char * temp = (*ifaces)[i];
		(*ifaces)[i] = (*ifaces)[(*n_ifaces) - i - 1];
		(*ifaces)[(*n_ifaces) - i - 1] = temp;
	}

	<<close ifupdown state>>
}
@

The last of these functions is used to modify a state file, specifically
for the interface given as the first argument.  If the second argument is
NULL then any existing state for that interface is removed from the file,
otherwise any existing state is changed to the new state or a new state
line is appended.

<<main functions>>=
static void update_state(const char *argv0, const char *iface, const char *state)
{
	FILE *tmp_fp;

	<<open ifupdown state>>

	if (no_act)
		goto noact;

	tmp_fp = fopen(tmpstatefile, "w");
	if (tmp_fp == NULL) {
		fprintf(stderr, 
			"%s: failed to open temporary statefile %s: %s\n",
			argv0, tmpstatefile, strerror(errno));
		exit (1);
	}

	while((p = fgets(buf, sizeof buf, state_fp)) != NULL) {
		<<parse ifupdown state line>>

		if (strncmp(iface, pch, strlen(iface)) == 0) {
			if (pch[strlen(iface)] == '=') {
				if (state != NULL) {
					fprintf (tmp_fp, "%s=%s\n",
						 iface, state);
					state = NULL;
				}

				continue;
			}
		}

		fprintf (tmp_fp, "%s\n", pch);
	}

	if (state != NULL)
		fprintf (tmp_fp, "%s=%s\n", iface, state);

	fclose (tmp_fp);
	if (rename (tmpstatefile, statefile)) {
		fprintf(stderr, 
			"%s: failed to overwrite statefile %s: %s\n",
			argv0, statefile, strerror(errno));
		exit (1);
	}

	<<close ifupdown state>>
}
@ 

The state file is opened and locked, blocking parallel updates:

<<main function declarations>>=
static int lock_fd (int fd);
@ 

<<main functions>>=
static int lock_fd (int fd) {
	struct flock lock;

	lock.l_type = F_WRLCK;
	lock.l_whence = SEEK_SET;
	lock.l_start = 0;
	lock.l_len = 0;

	if  (fcntl(fd, F_SETLKW, &lock) < 0) {
		return -1;
	}

	return 0;
}
@ 

<<open ifupdown state>>=
FILE *state_fp;
char buf[80];
char *p;

state_fp = fopen(statefile, no_act ? "r" : "a+");
if (state_fp == NULL) {
	if (!no_act) {
		fprintf(stderr, 
			"%s: failed to open statefile %s: %s\n",
			argv0, statefile, strerror(errno));
		exit (1);
	} else {
		goto noact;
	}
}

if (!no_act) {
	int flags;

	if ((flags = fcntl(fileno(state_fp), F_GETFD)) < 0
	    || fcntl(fileno(state_fp), F_SETFD, flags | FD_CLOEXEC) < 0) {
		fprintf(stderr, 
			"%s: failed to set FD_CLOEXEC on statefile %s: %s\n",
			argv0, statefile, strerror(errno));
		exit(1);
	}

	if (lock_fd (fileno(state_fp)) < 0) {
		fprintf(stderr, 
			"%s: failed to lock statefile %s: %s\n",
			argv0, statefile, strerror(errno));
		exit(1);
	}
}
@

<<parse ifupdown state line>>=
char *pch;

pch = buf + strlen(buf) - 1;
while(pch > buf && isspace(*pch)) pch--;
*(pch+1) = '\0';

pch = buf;
while(isspace(*pch)) pch++;
@ 

<<close ifupdown state>>=
noact:
if (state_fp != NULL) {
	fclose(state_fp);
	state_fp = NULL;
}
@

This leaves our two useful chunks. The first checks to ensure what we're
proposing to do is reasonable (ie, we're not downing an interface that's
not up, or uping one that's not down).

<<check ifupdown state (possibly [[continue]])>>=
{
	if (<<we're bringing interfaces up>>) {
		if (current_state != NULL) {
			if (!do_all) {
				fprintf(stderr, 
					"%s: interface %s already configured\n",
					argv[0], iface);
			}
			continue;
		}
	} else if (<<we are taking interfaces down>>) {
		if (current_state == NULL) {
			if (!do_all) {
				fprintf(stderr, "%s: interface %s not configured\n",
					argv[0], iface);
			}
			continue;
		}
		strncpy(liface, current_state, 80);
		liface[79] = 0;
	} else if (<<we are querying an interface config>>) {
		if (current_state != NULL) {
			strncpy(liface, current_state, 80);
			liface[79] = 0;
			run_mappings = 0;
		}

	} else if (!<<we are listing known interfaces>>
	           && !<<we are querying an interface config>>)
	{
		assert(0);
	}
}
@ 

And finally, we also need to be able to update the state as we bring
interfaces up and down.

<<update ifupdown state>>=
{
	if (<<we are bringing interfaces up>>) {
		if ((current_state == NULL) || (no_act)) {
			if (failed == 1) {
				printf("Failed to bring up %s.\n", liface);
				update_state (argv[0], iface, NULL);
			} else {
				update_state (argv[0], iface, liface);
			}
		} else {
			update_state (argv[0], iface, liface);
		}
	} else if (<<we are taking interfaces down>>) {
		update_state (argv[0], iface, NULL);
	} else if (!<<we are listing known interfaces>>
	           && !<<we are querying an interface config>>)
	{
		assert(0);
	}
}
@ 

\appendix

\section{Architecture-dependent functions}
\subsection{Common functions}
<<common functions declarations>>=
#include "header.h"

int execable(char *);
#define iface_is_link() (!_iface_has(ifd->real_iface, ":."))
#define iface_has(s) _iface_has(ifd->real_iface, (s))
#define iface_is_lo() ((!strcmp(ifd->logical_iface, LO_IFACE)) && (!no_loopback))
int _iface_has(char *, char *);
void cleanup_hwaddress(interface_defn *ifd, char **pparam, int argc, char ** argv);
void make_hex_address(interface_defn *ifd, char **pparam, int argc, char ** argv);
void compute_v4_addr(interface_defn *ifd, char **pparam, int argc, char ** argv);
void compute_v4_mask(interface_defn *ifd, char **pparam, int argc, char ** argv);
void compute_v4_broadcast(interface_defn *ifd, char **pparam, int argc, char ** argv);
void set_preferred_lft(interface_defn *ifd, char **pparam, int argc, char ** argv);
void get_token(interface_defn *ifd, char **pparam, int argc, char ** argv);
void to_decimal(interface_defn *ifd, char **pparam, int argc, char ** argv);
void map_value(interface_defn *ifd, char **pparam, int argc, char ** argv);
@ 

<<common functions implementations>>=
#ifdef __GNUC__
#define UNUSED __attribute__((unused))
#else
#define UNUSED
#endif

int _iface_has(char *iface, char *delims) {
	char _iface[80];
	strncpy(_iface, iface, sizeof(_iface));
	_iface[sizeof(_iface) - 1] = 0;
	strtok(_iface, delims);
	void * token = strtok(NULL, delims);
	return (token != NULL);
}

int execable(char *program) {
	struct stat buf;

	if (0 == stat(program, &buf)) {
		if (S_ISREG(buf.st_mode) && (S_IXUSR & buf.st_mode)) return 1;
	}
	return 0;
}

void cleanup_hwaddress(interface_defn *ifd UNUSED, char **pparam, int argc UNUSED, char ** argv UNUSED) {
	char *rest = *pparam;
		/* we're shrinking the text, so no realloc needed */
	char *space = strchr(rest, ' ');

	if (space == NULL)
		return;

	*space = '\0';
	if (strcasecmp(rest, "ether") == 0 ||
		strcasecmp(rest, "ax25") == 0 ||
		strcasecmp(rest, "ARCnet") == 0 ||
		strcasecmp(rest, "netrom") == 0)
	{
		/* found deprecated <class> attribute */
		memmove(rest, space+1, strlen(space+1)+1);
	} else {
		*space = ' ';
	}
}

void make_hex_address(interface_defn *ifd UNUSED, char **pparam, int argc UNUSED, char ** argv UNUSED)
{
	char addrcomp[4];
	int maxlen = strlen("0000:0000");

	int ret = sscanf(*pparam, "%3hhu.%3hhu.%3hhu.%3hhu",
		&addrcomp[0], &addrcomp[1], &addrcomp[2], &addrcomp[3]);

	if (ret != 4)
		return;

	*pparam = realloc(*pparam, maxlen + 1);
	if (*pparam == NULL) return;
	snprintf(*pparam, maxlen + 1, "%.2hhx%.2hhx:%.2hhx%.2hhx",
		addrcomp[0], addrcomp[1], addrcomp[2], addrcomp[3]);
}

#include <arpa/inet.h>

void compute_v4_addr(interface_defn *ifd UNUSED, char **pparam, int argc UNUSED, char ** argv UNUSED)
{
	char s[INET_ADDRSTRLEN * 2 + 2]; /* 2 is for slash and \0 */
	strncpy(s, *pparam, sizeof(s));
	s[sizeof(s) - 1] = 0;

	char * token = strtok(s, "/");
	if (!token) return;

	*pparam = realloc(*pparam, strlen(token) + 1);
	if (*pparam == NULL) return;
	strcpy(*pparam, token);
}

void compute_v4_mask(interface_defn *ifd UNUSED, char **pparam, int argc UNUSED, char ** argv UNUSED)
{
	char s[INET_ADDRSTRLEN * 2 + 2]; /* 2 is for slash and \0 */
	strncpy(s, *pparam, sizeof(s));
	s[sizeof(s) - 1] = 0;

	char * token = strtok(s, "/");
	if (!token) return;

	uint8_t addr[sizeof(struct in_addr)];
	struct in_addr mask;
	if (inet_pton(AF_INET, token, &addr) != 1) return;
	token = strtok(NULL, "/");
	int maskwidth = -1;
	if (!token) {
		if (addr[0] <= 127) {
		    maskwidth = 8;
		} else if ((addr[0] >= 128) && (addr[0] <= 191)) {
		    maskwidth = 16;
		} else if ((addr[0] >= 192) && (addr[0] <= 223)) {
		    maskwidth = 24;
		} else {
		    maskwidth = 32;
		}
	} else {
		switch (inet_pton(AF_INET, token, &mask)) {
			case -1:
				return;

			case 0:
				if (sscanf(token, "%d", &maskwidth) != 1) return;
		}
	}
	if (maskwidth != -1) {
		mask.s_addr = htonl(~((1L << (32 - maskwidth)) - 1));
	}

	if (inet_ntop(AF_INET, &mask, s, sizeof(s)) == NULL) return;
	*pparam = realloc(*pparam, strlen(s) + 1);
	if (*pparam == NULL) return;
	strcpy(*pparam, s);
}

void compute_v4_broadcast(interface_defn *ifd, char **pparam, int argc UNUSED, char ** argv UNUSED)
{
	/* If we don't get special value don't do anything */
	if (strcmp(*pparam, "+") && strcmp(*pparam, "-")) return;

	struct in_addr addr;
	struct in_addr mask;

	char * s = get_var("address", strlen("address"), ifd);
	if (!s) return;
	int r = inet_pton(AF_INET, s, &addr);
	free(s);
	if (r != 1) return;

	s = get_var("netmask", strlen("netmask"), ifd);
	if (!s) return;
	r = inet_pton(AF_INET, s, &mask);
	free(s);
	if (r != 1) return;

	if (mask.s_addr != htonl(0xfffffffe)) {
	    if (!strcmp(*pparam, "+")) {
		addr.s_addr |= ~mask.s_addr;
	    }

	    if (!strcmp(*pparam, "-")) {
		addr.s_addr &= mask.s_addr;
	    }
	} else {
	    if (!strcmp(*pparam, "+")) {
		addr.s_addr = 0xffffffff;
	    }

	    if (!strcmp(*pparam, "-")) {
		addr.s_addr = 0;
	    }
	}

	char buffer[INET_ADDRSTRLEN + 1];
	if (inet_ntop(AF_INET, &addr, buffer, sizeof(buffer)) == NULL) return;
	*pparam = realloc(*pparam, strlen(buffer) + 1);
	if (*pparam == NULL) return;
	strcpy(*pparam, buffer);
}

void set_preferred_lft(interface_defn *ifd, char **pparam, int argc UNUSED, char ** argv UNUSED)
{
	if (!ifd->real_iface) return;
	if (iface_has(":")) {
		char s[] = "0";
		*pparam = realloc(*pparam, sizeof(s));
		if (*pparam == NULL) return;
		strcpy(*pparam, s);
	}
}

void get_token(interface_defn *ifd UNUSED, char **pparam, int argc, char ** argv)
{
	if (argc == 0) return;

	int token_no;
	if (argc == 1) {
		token_no = 0;
	} else {
		token_no = atoi(argv[1]);
	}

	char * s = strdup(*pparam);
	char * token = strtok(s, argv[0]);
	while (token_no > 0) {
		token = strtok(NULL, argv[0]);
		token_no--;
	}
	if (token) {
	    strcpy(*pparam, token);
	} else {
	    if (argc == 3) {
		*pparam = realloc(*pparam, strlen(argv[2]) + 1);
		if (*pparam == NULL) return;
		strcpy(*pparam, argv[2]);
	    }
	}
	free(s);
}

void to_decimal(interface_defn *ifd UNUSED, char **pparam, int argc, char ** argv)
{
	int base = 10;

	if (argc == 1) {
		base = atoi(argv[0]);
	}

	char * result;
	long value = strtol(*pparam, &result, base);
	if (result == *pparam) return;

	snprintf(*pparam, strlen(*pparam) + 1, "%ld", value);
}

void map_value(interface_defn *ifd UNUSED, char **pparam, int argc, char ** argv)
{
	if (argc < 2) return;

	int value;
	if (argc == 2) {
		value = (atoi(*pparam) ||
			strcasecmp(*pparam, "on") == 0 ||
			strcasecmp(*pparam, "true") == 0 ||
			strcasecmp(*pparam, "yes") == 0);
	}
	if ((value < argc) && (argv[value] != NULL)) {
		*pparam = realloc(*pparam, strlen(argv[value]) + 1);
		if (*pparam == NULL) return;
		strcpy(*pparam, argv[value]);
	} else {
		*pparam = realloc(*pparam, 1);
		if (*pparam == NULL) return;
		*pparam[0] = 0;
	}
}

@ 

\subsection{Linux-specific functions}

<<archlinux.h>>=
unsigned int mylinuxver();
unsigned int mylinux(int,int,int);
<<common functions declarations>>
@

<<archlinux.c>>=
#include <stdio.h>
#include <stdlib.h>
#include <string.h>
#include <unistd.h>
#include <sys/utsname.h>
#include <sys/stat.h>

#include "archlinux.h"

unsigned int mylinuxver() {
	static int maj = -1, rev = 0, min = 0;

	if (maj == -1) {
		struct utsname u;
		char *pch;
		uname(&u);
		maj = atoi(u.release);
		pch = strchr(u.release, '.');
		if (pch) {
			rev = atoi(pch+1);
			pch = strchr(pch+1, '.');
			if (pch) {
				min = atoi(pch+1);
			}
		}
	}

	return mylinux(maj,rev,min);
}

unsigned int mylinux(int maj, int rev, int min) { 
	return min | rev << 10 | maj << 13;
}

<<common functions implementations>>
@ 

\subsection{kFreeBSD-specific functions}

<<archkfreebsd.h>>=
/* no OS-specific functions yet */
<<common functions declarations>>
@

<<archkfreebsd.c>>=
#include <stdio.h>
#include <stdlib.h>
#include <string.h>
#include <unistd.h>
#include <sys/utsname.h>
#include <sys/stat.h>

#include "archkfreebsd.h"

<<common functions implementations>>
@ 

\subsection{Hurd-specific functions}

<<archhurd.h>>=
/* no OS-specific functions yet */
<<common functions declarations>>
@

<<archhurd.c>>=
#include <stdio.h>
#include <stdlib.h>
#include <string.h>
#include <unistd.h>
#include <sys/utsname.h>
#include <sys/stat.h>

#include "archhurd.h"

<<common functions implementations>>
@ 

\section{Linux Address Families}
\subsection{IPv4 Address Family}

<<address family declarations>>=
extern address_family addr_inet;
@ 

<<address family references>>=
&addr_inet, 
@ 

<<inet.defn>>=
address_family inet
architecture linux

<<Linux inet methods>>

@ 

<<Linux inet methods>>=
<<Linux inet methods: loopback>>

<<Linux inet methods: static>>

<<inet methods: manual>>

<<Linux inet methods: dhcp>>

<<inet methods: bootp>>

<<Linux inet methods: tunnel>>

<<inet methods: ppp>>

<<inet methods: wvdial>>

<<inet methods: ipv4ll>>
@ 

<<Linux inet methods: loopback>>=
method loopback
  description
    This method may be used to define the IPv4 loopback interface.

  up
    ip link set dev %iface% up if (!iface_is_lo())

  down
    ip link set dev %iface% down if (!iface_is_lo())
@ 

<<Linux inet methods: static>>=
method static
  description
    This method may be used to define Ethernet interfaces with statically
    allocated IPv4 addresses.
      
  options
    address address             -- Address (dotted quad/netmask) *required*
    netmask mask                -- Netmask (dotted quad or CIDR)
    broadcast broadcast_address -- Broadcast address (dotted quad, + or -) [+]
    metric metric               -- Routing metric for default gateway (integer)
    gateway address             -- Default gateway (dotted quad)
    pointopoint address         -- Address of other end point (dotted quad). \
                                   Note the spelling of "point-to".
    hwaddress address           -- Link local address.
    mtu size                    -- MTU size
    scope                       -- Address validity scope. Possible values: \
                                   global, link, host

  conversion
    hwaddress cleanup_hwaddress
    address compute_v4_mask =netmask?
    address compute_v4_addr
    broadcast compute_v4_broadcast

  up
    ip addr add %address%[[/%netmask%]] [[broadcast %broadcast%]] \
	[[peer %pointopoint%]] [[scope %scope%]] dev %iface% label %iface%
    ip link set dev %iface% [[mtu %mtu%]] [[address %hwaddress%]] up

    [[ ip route add default via %gateway% [[metric %metric%]] dev %iface% ]]

  down
    [[ ip route del default via %gateway% [[metric %metric%]] dev %iface% 2>&1 1>/dev/null || true ]]
    ip -4 addr flush dev %iface% label %iface%
    ip link set dev %iface% down \
		if (iface_is_link())
@

<<inet methods: manual>>=
method manual
  description
    This method may be used to define interfaces for which no configuration
    is done by default.  Such interfaces can be configured manually by
    means of *up* and *down* commands or /etc/network/if-*.d scripts.

  up

  down
@ 

<<Linux inet methods: dhcp>>=
method dhcp
  description
    This method may be used to obtain an address via DHCP with any of
    the tools: dhclient, pump, udhcpc, dhcpcd.
    (They have been listed in their order of precedence.)
    If you have a complicated DHCP setup you should
    note that some of these clients use their own configuration files
    and do not obtain their configuration information via *ifup*.

  options
    hostname hostname       -- Hostname to be requested (pump, dhcpcd, udhcpc)
    metric metric           -- Metric for added routes (dhclient)
    leasehours leasehours   -- Preferred lease time in hours (pump)
    leasetime leasetime     -- Preferred lease time in seconds (dhcpcd)
    vendor vendor           -- Vendor class identifier (dhcpcd)
    client client           -- Client identifier (dhcpcd, udhcpc)
    hwaddress address       -- Hardware address.

  conversion
    hwaddress cleanup_hwaddress

  up
    [[ip link set dev %iface% address %hwaddress%]]
    dhclient -v -pf /run/dhclient.%iface%.pid -lf /var/lib/dhcp/dhclient.%iface%.leases %iface% \
	[[-e IF_METRIC=%metric%]] \
        if (execable("/sbin/dhclient"))
    dhclient3 -pf /run/dhclient.%iface%.pid -lf /var/lib/dhcp3/dhclient.%iface%.leases %iface% \
	[[-e IF_METRIC=%metric%]] \
        elsif (execable("/sbin/dhclient3"))
    pump -i %iface% [[-h %hostname%]] [[-l %leasehours%]] \
        elsif (execable("/sbin/pump") && mylinuxver() >= mylinux(2,1,100))
    udhcpc -n -p /run/udhcpc.%iface%.pid -i %iface% [[-H %hostname%]] \
           [[-c %client%]] \
        elsif (execable("/sbin/udhcpc") && mylinuxver() >= mylinux(2,2,0))
    dhcpcd [[-h %hostname%]] [[-i %vendor%]] [[-I %client%]] \
           [[-l %leasetime%]] %iface% \
        elsif (execable("/sbin/dhcpcd"))

  down
    dhclient -v -r -pf /run/dhclient.%iface%.pid -lf /var/lib/dhcp/dhclient.%iface%.leases %iface% \
        if (execable("/sbin/dhclient"))
    dhclient3 -r -pf /run/dhclient.%iface%.pid -lf /var/lib/dhcp3/dhclient.%iface%.leases %iface% \
        elsif (execable("/sbin/dhclient3"))
    pump -i %iface% -r \
        elsif (execable("/sbin/pump") && mylinuxver() >= mylinux(2,1,100))
    kill -USR2 $(cat /run/udhcpc.%iface%.pid); kill -TERM $(cat /run/udhcpc.%iface%.pid) \
        elsif (execable("/sbin/udhcpc"))
    dhcpcd -k %iface% \
        elsif (execable("/sbin/dhcpcd"))

    ip link set dev %iface% down \
		if (iface_is_link())
@ 

<<inet methods: bootp>>=
method bootp
  description
    This method may be used to obtain an address via bootp.

  options
    bootfile file  -- Tell the server to use /file/ as the bootfile.
    server address -- Use the IP address /address/ to communicate with \
                      the server.
    hwaddr addr    -- Use /addr/ as the hardware address instead of \
                      whatever it really is.

  up
    bootpc [[--bootfile %bootfile%]] --dev %iface% [[--server %server%]] \
           [[--hwaddr %hwaddr%]] --returniffail --serverbcast

  down
    ip link set dev %iface% down \
        if (execable("/sbin/ip") && iface_is_link())
    ifconfig %iface% down \
        elsif (1)
@ 

<<Linux inet methods: tunnel>>=
method tunnel
  description
    This method is used to create GRE or IPIP tunnels. You need to have
    the *ip* binary from the *iproute* package. For GRE tunnels, you
    will need to load the ip_gre module and the ipip module for
    IPIP tunnels.
  options
    address address       -- Local address (dotted quad) *required*
    mode type             -- Tunnel type (either GRE or IPIP) *required*
    endpoint address      -- Address of other tunnel endpoint *required*
    dstaddr address       -- Remote address (remote address inside tunnel)
    local address         -- Address of the local endpoint
    gateway address       -- Default gateway
    ttl time              -- TTL setting
    mtu size              -- MTU size
  up
    ip tunnel add %iface% mode %mode% remote %endpoint% [[local %local%]] \
       [[ttl %ttl%]]
    ip link set %iface% up [[mtu %mtu%]]
    ip addr add %address%/%netmask% dev %iface% [[peer %dstaddr%]]
    [[ ip route add default via %gateway% ]]
  down
    ip tunnel del %iface%
@ 

<<inet methods: ppp>>=
method ppp
  description
    This method uses pon/poff to configure a PPP interface. See those
    commands for details.
  options
    provider name  -- Use /name/ as the provider (from /etc/ppp/peers).
    unit number    -- Use /number/ as the ppp unit number.
    options string -- Pass /string/ as additional options to pon.
  up
    pon [[%provider%]] [[unit %unit%]] [[%options%]]
  down
    poff [[%provider%]]
@ 

<<inet methods: wvdial>>=
method wvdial
  description
    This method uses wvdial to configure a PPP interface. See that command
    for more details.
  options
    provider name  -- Use /name/ as the provider (from /etc/wvdial.conf).
  up
    /sbin/start-stop-daemon --start -x /usr/bin/wvdial \
                      -p /run/wvdial.%iface%.pid -b -m -- [[ %provider% ]]
  down
    /sbin/start-stop-daemon --stop -x /usr/bin/wvdial \
                      -p /run/wvdial.%iface%.pid -s 2

<<inet methods: ipv4ll>>=
method ipv4ll
  description
    This method uses avahi-autoipd to configure an interface with an
    IPv4 Link-Layer address (169.254.0.0/16 family). This method is also
    known as APIPA or IPAC, and often colloquially referred to
    as "Zeroconf address".
  up
    /usr/sbin/avahi-autoipd -D %iface%
  down
    /usr/sbin/avahi-autoipd --kill %iface%
@




\subsection{IPv6 Address Family}

<<address family declarations>>=
extern address_family addr_inet6;
@ 

<<address family references>>=
&addr_inet6,
@ 

<<inet6.defn>>=
address_family inet6
architecture linux

method auto
  description
    This method may be used to define interfaces with automatically assigned
    IPv6 addresses. Using this method on its own doesn't mean that RDNSS options
    will be applied, too. To make this happen, *rdnssd* daemon must be installed,
    properly configured and running.
    If stateless DHCPv6 support is turned on, then additional network
    configuration parameters such as DNS and NTP servers will be retrieved
    from a DHCP server. Please note that on ifdown, the lease is not currently
    released (a known bug).

  options
    privext int            -- Privacy extensions (RFC3041) (0=off, 1=assign, 2=prefer)
    dhcp int               -- Use stateless DHCPv6 (0=off, 1=on)
  up
    modprobe -q net-pf-10 > /dev/null 2>&1 || true # ignore failure.
    [[sysctl -q -e -w net.ipv6.conf.%iface/.//%.use_tempaddr=%privext%]]
    sysctl -q -e -w net.ipv6.conf.%iface/.//%.accept_ra=1
    sysctl -q -e -w net.ipv6.conf.%iface/.//%.autoconf=1
    ip link set dev %iface% up
    dhclient -6 -S -pf /run/dhclient6.%iface%.pid -lf /var/lib/dhcp/dhclient6.%iface%.leases %iface% \
        if (var_true("dhcp", ifd) && execable("/sbin/dhclient"))
  down
    ip -6 addr flush dev %iface% scope global
    ip link set dev %iface% down \
		if (iface_is_link())

method loopback
  description
    This method may be used to define the IPv6 loopback interface.
  up
    -ip link set dev %iface% up 2>/dev/null if (!iface_is_lo())
    -ip addr add dev %iface% ::1 2>/dev/null if (!iface_is_lo())
  down
    -ip addr del dev %iface% ::1 2>/dev/null if (!iface_is_lo())
    -ip link set dev %iface% down 2>/dev/null if (!iface_is_lo())

method static
  description
    This method may be used to define interfaces with statically assigned
    IPv6 addresses. By default, stateless autoconfiguration is disabled for
    this interface.

  options
    address address        -- Address (colon delimited/netmask) *required*
    netmask mask           -- Netmask (number of bits, eg 64)
    gateway address        -- Default gateway (colon delimited)
    media type             -- Medium type, driver dependent
    hwaddress address      -- Hardware address
    mtu size               -- MTU size
    accept_ra int          -- Accept router advertisements (0=off, 1=on)
    autoconf int           -- Perform stateless autoconfiguration (0=off, 1=on) [0]
    privext int            -- Privacy extensions (RFC3041) (0=off, 1=assign, 2=prefer)
    scope                  -- Address validity scope. Possible values: \
                              global, site, link, host
    preferred-lifetime int -- Time that address remains preferred []

  conversion
    hwaddress cleanup_hwaddress
    preferred-lifetime set_preferred_lft
    address (get_token / 1 "") =netmask?
    address (get_token / 0 "")

  up
    modprobe -q net-pf-10 > /dev/null 2>&1 || true # ignore failure.
    [[sysctl -q -e -w net.ipv6.conf.%iface/.//%.use_tempaddr=%privext%]]
    [[sysctl -q -e -w net.ipv6.conf.%iface/.//%.accept_ra=%accept_ra%]]
    [[sysctl -q -e -w net.ipv6.conf.%iface/.//%.autoconf=%autoconf%]]
    ip link set dev %iface% [[mtu %mtu%]] [[address %hwaddress%]] up
    ip -6 addr add %address%[[/%netmask%]] [[scope %scope%]] dev %iface% [[preferred_lft %preferred-lifetime%]]
    [[ ip -6 route add default via %gateway% dev %iface% ]]

  down
    ip -6 addr flush dev %iface% scope global
    ip link set dev %iface% down \
		if (iface_is_link())

method manual
  description
    This method may be used to define interfaces for which no configuration
    is done by default.  Such interfaces can be configured manually by
    means of *up* and *down* commands or /etc/network/if-*.d scripts.

  up

  down

method dhcp
  description
    This method may be used to obtain network interface configuration via
    stateful DHCPv6 with dhclient.  In stateful DHCPv6, the DHCP server is
    responsible for assigning addresses to clients.

  options
    hwaddress address      -- Hardware address
    accept_ra int          -- Accept router advertisements (0=off, 1=on) [0]
    autoconf int           -- Perform stateless autoconfiguration (0=off, 1=on)

  conversion
    hwaddress cleanup_hwaddress

  up
    modprobe -q net-pf-10 > /dev/null 2>&1 || true # ignore failure.
    [[sysctl -q -e -w net.ipv6.conf.%iface/.//%.accept_ra=%accept_ra%]]
    [[sysctl -q -e -w net.ipv6.conf.%iface/.//%.autoconf=%autoconf%]]
    ip link set dev %iface% [[address %hwaddress%]] up
    dhclient -6 -pf /run/dhclient6.%iface%.pid -lf /var/lib/dhcp/dhclient6.%iface%.leases %iface% \
        if (execable("/sbin/dhclient"))

  down
    dhclient -6 -r -pf /run/dhclient6.%iface%.pid -lf /var/lib/dhcp/dhclient6.%iface%.leases %iface% \
        if (execable("/sbin/dhclient"))
    ip link set dev %iface% down \
		if (iface_is_link())

method v4tunnel
  description
    This method may be used to setup an IPv6-over-IPv4 tunnel. It requires
    the *ip* command from the *iproute* package.

  options
    address address       -- Address (colon delimited) *required*
    netmask mask          -- Netmask (number of bits, eg 64) 
    endpoint address      -- Address of other tunnel endpoint (IPv4 \
                             dotted quad) *required*
    local address         -- Address of the local endpoint (IPv4 \
                             dotted quad)
    gateway address       -- Default gateway (colon delimited)
    ttl time              -- TTL setting
    mtu size              -- MTU size

  up
    modprobe -q net-pf-10 > /dev/null 2>&1 || true # ignore failure.
    ip tunnel add %iface% mode sit remote %endpoint% [[local %local%]] \
       [[ttl %ttl%]]
    ip link set %iface% up [[mtu %mtu%]]
    [[ ip addr add %address%[[/%netmask%]] dev %iface% ]]
    [[ ip route add %gateway% dev %iface% ]]
    [[ ip route add ::/0 via %gateway% dev %iface% ]]

  down
    ip tunnel del %iface%

method 6to4
  description
    This method may be used to setup an 6to4 tunnel. It requires
    the *ip* command from the *iproute* package.

  options
    local address         -- Address of the local endpoint (IPv4 \
                             dotted quad) *required*
    ttl time              -- TTL setting
    mtu size              -- MTU size

  conversion
    local make_hex_address =hexaddress

  up
    modprobe -q net-pf-10 > /dev/null 2>&1 || true # ignore failure.
    ip tunnel add %iface% mode sit remote any local %local% \
       [[ttl %ttl%]]
    ip link set %iface% up [[mtu %mtu%]]
    ip addr add 2002:%hexaddress%::1/16 dev %iface%
    ip route add 2000::/3 via ::192.88.99.1 dev %iface%

  down
    ip -6 route flush dev %iface%
    ip link set dev %iface% down
    ip tunnel del %iface%
@ 

\subsection{IPX Address Family}

<<address family declarations>>=
extern address_family addr_ipx;
@ 

<<address family references>>=
&addr_ipx,
@ 

<<ipx.defn>>=
address_family ipx
architecture linux

method static
  description
    This method may be used to setup an IPX interface.  It requires the
    /ipx_interface/ command.

  options
    frame type             -- /type/ of Ethernet frames to use (e.g. *802.2*)
    netnum id              -- Network number

  up
    ipx_interface add %iface% %frame% %netnum%

  down
    ipx_interface del %iface% %frame%

method dynamic
  description
    This method may be used to setup an IPX interface dynamically.

  options
    frame type             -- /type/ of Ethernet frames to use (e.g. *802.2*)

  up
    ipx_interface add %iface% %frame%

  down
    ipx_interface del %iface% %frame%
@ 

\subsection{CAN Address Family}

<<address family declarations>>=
extern address_family addr_can;
@ 

<<address family references>>=
&addr_can,
@ 

<<can.defn>>=
address_family can
architecture linux

method static
  description
    This method may be used to setup an Controller Area Network (CAN)
    interface. It requires the the *ip* command from the *iproute* package.

  options
    bitrate bitrate		-- bitrate (1..1000000) *required*
    samplepoint samplepoint	-- sample point (0.000..0.999)
    loopback loopback		-- loop back CAN Messages (on|off)
    listenonly listenonly	-- listen only mode (on|off)
    triple triple		-- activate triple sampling (on|off)
    oneshot oneshot		-- one shot mode (on|off)
    berr berr			-- activate berr reporting (on|off) 

  up
    ip link set %iface% type can bitrate %bitrate%
    [[ ip link set %iface% type can loopback %loopback% ]]
    [[ ip link set %iface% type can listen-only %listenonly% ]]
    [[ ip link set %iface% type can triple-sampling %triple% ]]
    [[ ip link set %iface% type can one-shot %oneshot% ]]
    [[ ip link set %iface% type can berr-reporting %berr% ]]
    ip link set %iface% up

  down
    ip link set %iface% down
@

\section{kFreeBSD Address Families}
\subsection{IPv4 Address Family}

<<inet.defn>>=
architecture kfreebsd

<<kFreeBSD inet methods>>
@ 

<<kFreeBSD inet methods>>=
<<kFreeBSD inet methods: loopback>>

<<kFreeBSD inet methods: static>>

<<inet methods: manual>>

<<kFreeBSD inet methods: dhcp>>

<<inet methods: bootp>>

<<inet methods: ppp>>

<<inet methods: wvdial>>

<<inet methods: ipv4ll>>
@ 

<<kFreeBSD inet methods: loopback>>=
method loopback
  description
    This method may be used to define the IPv4 loopback interface.

  up
    ifconfig %iface% 127.0.0.1 up \
	if (!iface_is_lo())

  down
    ifconfig %iface% down \
	if (!iface_is_lo())
@ 

<<kFreeBSD inet methods: static>>=
method static
  description
    This method may be used to define Ethernet interfaces with statically
    allocated IPv4 addresses.
      
  options
    address address             -- Address (dotted quad/netmask) *required*
    netmask mask                -- Netmask (dotted quad or CIDR)
    broadcast broadcast_address -- Broadcast address (dotted quad)
    metric metric               -- Routing metric for default gateway (integer)
    gateway address             -- Default gateway (dotted quad)
    pointopoint address         -- Address of other end point (dotted quad). \
                                   Note the spelling of "point-to".
    hwaddress address           -- Link local address.
    mtu size                    -- MTU size

  conversion
    hwaddress cleanup_hwaddress

  up
    [[ ifconfig %iface% link %hwaddress%]]
    ifconfig %iface% %address% [[netmask %netmask%]] [[broadcast %broadcast%]] \
	[[pointopoint %pointopoint%]] [[media %media%]] [[mtu %mtu%]] \
	up
    [[ /lib/freebsd/route add default %gateway% ]]

  down
    [[ /lib/freebsd/route del default %gateway% 2>&1 1>/dev/null || true ]]
    ifconfig %iface% down
@

<<kFreeBSD inet methods: dhcp>>=
method dhcp
  description
    This method may be used to obtain an address via DHCP with any of
    the tools: dhclient, udhcpc, dhcpcd.
    (They have been listed in their order of precedence.)
    If you have a complicated DHCP setup you should
    note that some of these clients use their own configuration files
    and do not obtain their configuration information via *ifup*.

  options
    hostname hostname       -- Hostname to be requested (dhcpcd, udhcpc)
    metric metric           -- Metric for added routes (dhclient)
    leasetime leasetime     -- Preferred lease time in seconds (dhcpcd)
    vendor vendor           -- Vendor class identifier (dhcpcd)
    client client           -- Client identifier (dhcpcd, udhcpc)
    hwaddress address       -- Hardware Address.

  conversion
    hwaddress cleanup_hwaddress

  up
    [[ifconfig %iface% link %hwaddress%]]
    dhclient -v -pf /run/dhclient.%iface%.pid -lf /var/lib/dhcp/dhclient.%iface%.leases %iface% \
	[[-e IF_METRIC=%metric%]] \
        if (execable("/sbin/dhclient"))
    dhclient3 -pf /run/dhclient.%iface%.pid -lf /var/lib/dhcp3/dhclient.%iface%.leases %iface% \
	[[-e IF_METRIC=%metric%]] \
        elsif (execable("/sbin/dhclient3"))
    udhcpc -n -p /run/udhcpc.%iface%.pid -i %iface% [[-H %hostname%]] \
           [[-c %client%]] \
        elsif (execable("/sbin/udhcpc"))
    dhcpcd [[-h %hostname%]] [[-i %vendor%]] [[-I %client%]] \
           [[-l %leasetime%]] %iface% \
        elsif (execable("/sbin/dhcpcd"))

  down
    dhclient -v -r -pf /run/dhclient.%iface%.pid -lf /var/lib/dhcp/dhclient.%iface%.leases %iface% \
        if (execable("/sbin/dhclient"))
    dhclient3 -r -pf /run/dhclient.%iface%.pid -lf /var/lib/dhcp3/dhclient.%iface%.leases %iface% \
        elsif (execable("/sbin/dhclient3"))
    kill -USR2 $(cat /run/udhcpc.%iface%.pid); kill -TERM $(cat /run/udhcpc.%iface%.pid) \
        elsif (execable("/sbin/udhcpc"))
    dhcpcd -k %iface% \
        elsif (execable("/sbin/dhcpcd"))

    ifconfig %iface% down
@ 

\subsection{IPv6 Address Family}

<<inet6.defn>>=
architecture kfreebsd

method loopback
  description
    This method may be used to define the IPv6 loopback interface.
  up
    ifconfig %iface% inet6 ::1 \
	if (!iface_is_lo())
  down
    ifconfig %iface% down \
	if (!iface_is_lo())

method static
  description
    This method may be used to define interfaces with statically assigned
    IPv6 addresses.

  options
    address address        -- Address (colon delimited) *required*
    netmask mask           -- Netmask (number of bits, eg 64) *required*
    gateway address        -- Default gateway (colon delimited)
    media type             -- Medium type, driver dependent
    hwaddress address      -- Hardware address
    mtu size               -- MTU size

  conversion
    hwaddress cleanup_hwaddress

  up
    ifconfig %iface% [[media %media%]] [[link %hwaddress%]] [[mtu %mtu%]] up
    ifconfig %iface% inet6 %address%/%netmask% alias
    [[ /lib/freebsd/route add -inet6 ::/0 %gateway% ]]

  down
    [[ /lib/freebsd/route -n del -inet6 ::/0 2>&1 1>/dev/null || true ]]
    [[ ifconfig %iface% inet6 %address% -alias ]]
    ifconfig %iface% down

method manual
  description
    This method may be used to define interfaces for which no configuration
    is done by default.  Such interfaces can be configured manually by
    means of *up* and *down* commands or /etc/network/if-*.d scripts.

  up

  down

method auto
  description
    This method may be used to define interfaces with automatically assigned
    IPv6 addresses. Using this method on its own doesn't mean that RDNSS options
    will be applied, too. To make this happen, *rdnssd* daemon must be installed,
    properly configured and running.
    If stateless DHCPv6 support is turned on, then additional network
    configuration parameters such as DNS and NTP servers will be retrieved
    from a DHCP server. Please note that on ifdown, the lease is not currently
    released (a known bug).

  options
    dhcp int               -- Use stateless DHCPv6 (0=off, 1=on)

  conversion
    hwaddress cleanup_hwaddress

  up
    sysctl -q -e -w net.inet6.ip6.accept_rtadv=1
    ifconfig %iface% up
    dhclient -6 -S -pf /run/dhclient6.%iface%.pid -lf /var/lib/dhcp/dhclient6.%iface%.leases %iface% \
        if (var_true("dhcp", ifd) && execable("/sbin/dhclient"))

  down
    ifconfig %iface% down



method dhcp
  description
    This method may be used to obtain network interface configuration via
    stateful DHCPv6 with dhclient.  In stateful DHCPv6, the DHCP server is
    responsible for assigning addresses to clients.

  options
    hwaddress address      -- Hardware address

  conversion
    hwaddress cleanup_hwaddress

  up
    ifconfig %iface% [[link %hwaddress%]] up
    dhclient -6 -pf /run/dhclient6.%iface%.pid -lf /var/lib/dhcp/dhclient6.%iface%.leases %iface% \
        if (execable("/sbin/dhclient"))

  down
    dhclient -6 -r -pf /run/dhclient6.%iface%.pid -lf /var/lib/dhcp/dhclient6.%iface%.leases %iface% \
        if (execable("/sbin/dhclient"))
    ifconfig %iface% down

@ 

\section{Hurd Address Families}
\subsection{IPv4 Address Family}

<<inet.defn>>=
architecture hurd

<<Hurd inet methods>>
@ 

<<Hurd inet methods>>=
<<Hurd inet methods: loopback>>

<<Hurd inet methods: static>>

<<inet methods: manual>>

<<Hurd inet methods: dhcp>>

<<inet methods: bootp>>

<<inet methods: ppp>>

<<inet methods: wvdial>>

<<inet methods: ipv4ll>>
@ 

<<Hurd inet methods: loopback>>=
method loopback
  description
    This method may be used to define the IPv4 loopback interface.

  up
    inetutils-ifconfig --interface %iface% 127.0.0.1 --up \
	if (!iface_is_lo())

  down
    inetutils-ifconfig --interface %iface% --down \
	if (!iface_is_lo())
@ 

<<Hurd inet methods: static>>=
method static
  description
    This method may be used to define Ethernet interfaces with statically
    allocated IPv4 addresses.
      
  options
    address address             -- Address (dotted quad/netmask) *required*
    netmask mask                -- Netmask (dotted quad or CIDR)
    broadcast broadcast_address -- Broadcast address (dotted quad)
    metric metric               -- Routing metric for default gateway (integer)
    gateway address             -- Default gateway (dotted quad)
    pointopoint address         -- Address of other end point (dotted quad). \
                                   Note the spelling of "point-to".
    hwaddress address           -- Link local address (Not yet supported)
    mtu size                    -- MTU size

  conversion
    hwaddress cleanup_hwaddress

  up
    [[Warning: Option hwaddress: %hwaddress% not yet supported]]
    inetutils-ifconfig --interface %iface% --address %address% [[--netmask %netmask%]] \
    [[--broadcast %broadcast%]] [[--mtu %mtu%]] --up
    [[fsysopts /servers/socket/2 $(showtrans /servers/socket/2) --gateway %gateway% ]]

  down
    inetutils-ifconfig --interface %iface% --down
@

<<Hurd inet methods: dhcp>>=
method dhcp
  description
    This method may be used to obtain an address via DHCP with any of
    the tools: dhclient, udhcpc, dhcpcd.
    (They have been listed in their order of precedence.)
    If you have a complicated DHCP setup you should
    note that some of these clients use their own configuration files
    and do not obtain their configuration information via *ifup*.

  options
    hostname hostname       -- Hostname to be requested (dhcpcd, udhcpc)
    leasetime leasetime     -- Preferred lease time in seconds (dhcpcd)
    vendor vendor           -- Vendor class identifier (dhcpcd)
    client client           -- Client identifier (dhcpcd, udhcpc)
    hwaddress address       -- Hardware Address (Not yet supported)

  conversion
    hwaddress cleanup_hwaddress

  up
    [[Warning: Option hwaddress: %hwaddress% not yet supported]]
    dhclient -v -pf /run/dhclient.%iface%.pid -lf /var/lib/dhcp/dhclient.%iface%.leases %iface% \
        if (execable("/sbin/dhclient"))
    udhcpc -n -p /run/udhcpc.%iface%.pid -i %iface% [[-H %hostname%]] \
           [[-c %client%]] \
        elsif (execable("/sbin/udhcpc"))
    dhcpcd [[-h %hostname%]] [[-i %vendor%]] [[-I %client%]] \
           [[-l %leasetime%]] %iface% \
        elsif (execable("/sbin/dhcpcd"))

  down
    dhclient -v -r -pf /run/dhclient.%iface%.pid -lf /var/lib/dhcp/dhclient.%iface%.leases %iface% \
        if (execable("/sbin/dhclient"))
    kill -USR2 $(cat /run/udhcpc.%iface%.pid); kill -TERM $(cat /run/udhcpc.%iface%.pid) \
        elsif (execable("/sbin/udhcpc"))
    dhcpcd -k %iface% \
        elsif (execable("/sbin/dhcpcd"))

    ifconfig --interface %iface% --down
@ 

\subsection{IPv6 Address Family}

<<inet6.defn>>=
architecture hurd

method loopback
  description
    This method may be used to define the IPv6 loopback interface.
  up
    [[FIXME: Add proper commands here for ipv6]]
  down
    [[FIXME: Add proper commands here for ipv6]]

method static
  description
    This method may be used to define interfaces with statically assigned
    IPv6 addresses.

  options
    address address        -- Address (colon delimited) *required*
    netmask mask           -- Netmask (number of bits, eg 64) *required*
    gateway address        -- Default gateway (colon delimited)
    media type             -- Medium type, driver dependent
    hwaddress address      -- Hardware address  (Not yet supported)
    mtu size               -- MTU size

  conversion
    hwaddress cleanup_hwaddress

  up
    [[FIXME: Add proper commands here for ipv6]]
    [[Warning: Option media: %media% not yet supported]]
    [[Warning: Option hwaddress: %hwaddress% not yet supported]]

  down
    [[FIXME: Add proper commands here for ipv6]]

method manual
  description
    This method may be used to define interfaces for which no configuration
    is done by default.  Such interfaces can be configured manually by
    means of *up* and *down* commands or /etc/network/if-*.d scripts.

  up

  down

method dhcp
  description
    This method may be used to obtain network interface configuration via
    stateful DHCPv6 with dhclient.  In stateful DHCPv6, the DHCP server is
    responsible for assigning addresses to clients.

  options
    hwaddress address      -- Hardware address (Not yet supported)

  conversion
    hwaddress cleanup_hwaddress

  up
    [[Warning: Option hwaddress: %hwaddress% not yet supported]]
    inetutils-ifconfig --interface %iface% --up
    dhclient -6 -pf /run/dhclient6.%iface%.pid -lf /var/lib/dhcp/dhclient6.%iface%.leases %iface% \
        if (execable("/sbin/dhclient"))

  down
    dhclient -6 -r -pf /run/dhclient6.%iface%.pid -lf /var/lib/dhcp/dhclient6.%iface%.leases %iface% \
        if (execable("/sbin/dhclient"))
    inetutils-ifconfig --interface %iface% --down

@ 

\section{Internal address metafamily}

<<address family declarations>>=
extern address_family addr_meta;
@ 

<<address family references>>=
&addr_meta, 
@ 

<<meta.defn>>=
address_family meta
architecture any

method none
	description
	up
	down
@ 

\section{Link pseudo address family}

<<exported symbols>>=
extern address_family addr_link;
@ 

<<link.defn>>=
address_family link
architecture linux

method none
  description
  conversion
    iface (get_token . 0 "") =link
    iface (get_token . 1 "") =vlan_id0
    iface (get_token : 0 "") =iface0
    vlan_id0 (get_token : 0 "") =vlan_id1
    vlan_id1 (to_decimal 10) =vlan_id
  up
    if test -d /sys/class/net/%link% -a \
            ! -d /sys/class/net/%iface0% ; \
    then \
        ip link set up dev %link%; \
        ip link add link %link% name %iface0% type vlan id %vlan_id%; \
    fi if (iface_has(".") && (!var_set_anywhere("bridge_ports", ifd)))
    -ip link set up dev %iface% 2>/dev/null \
	if (iface_is_lo())
  down
    ip link del %iface% if (iface_has(".") && (!var_set_anywhere("bridge_ports", ifd)))
    -ip link set down dev %iface% 2>/dev/null \
	if (iface_is_lo())

architecture kfreebsd

method none
  description
  up
    -ifconfig %iface% 127.0.0.1 up \
	if (iface_is_lo())
    -ifconfig %iface% inet6 ::1 \
	if (iface_is_lo())
  down
    -ifconfig %iface% down \
	if (iface_is_lo())

architecture hurd

method none
  description
  up
    -inetutils-ifconfig --interface %iface% 127.0.0.1 --up \
	if (iface_is_lo())
  down
    -inetutils-ifconfig --interface %iface% --down \
	if (iface_is_lo())

@ 

\begin{flushleft}
\bibliography{biblio}
\bibliographystyle{unsrt}
\end{flushleft}

\end{document}
% vim: noet ts=8
